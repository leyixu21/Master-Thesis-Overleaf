\documentclass{article}

% Language setting
% Replace `english' with e.g., `spanish' to change the document language
\usepackage[english]{babel}


% Set page size and margins
% Replace `letterpaper' with `a4paper' for UK/EU standard size
% 
\usepackage[a4paper, left=1.4in, right=1.4in, top=1.2in, bottom=1.2in]{geometry}

% Useful packages
\usepackage{amsmath}
\usepackage{graphicx}
\usepackage[colorlinks=true, allcolors=blue]{hyperref}
\usepackage{apacite}
\usepackage[acronym]{glossaries}
\usepackage[nottoc]{tocbibind}
\usepackage{natbib}
\glstoctrue
\usepackage[⟨options⟩]{fancyhdr}
\usepackage{parskip}
\usepackage{adjustbox}
\usepackage{multirow}
\usepackage{threeparttable}
\usepackage{chngcntr}
\counterwithin{figure}{section}
\counterwithin{table}{section}
\usepackage{titlesec}
\newcommand{\subsubsubsection}[1]{\paragraph{#1}\mbox{}\\}
\setcounter{secnumdepth}{4}
\setcounter{tocdepth}{4}
\usepackage{subcaption}
\usepackage[export]{adjustbox}
\usepackage{wrapfig}
\usepackage{amsthm}
\usepackage{subcaption}


\theoremstyle{definition}
\newtheorem{definition}{Definition}[section]

\theoremstyle{remark}
\newtheorem*{remark}{Remark}

\title{Discovering City Perception by Mining Semantic Trajectory}
\author{Leyi Xu}

\makeglossaries
\newacronym{gcd}{GCD}{Greatest Common Divisor}
\newacronym{lcm}{LCM}{Least Common Multiple}
\newacronym{ugc}{UGC}{User-Generated Content}
\newacronym{nlp}{NLP}{Natural Language Processing}
\newacronym{msm}{MSM}{Multidimensional Similarity Measure}
\newacronym{muitas}{MUITAS}{Multiple-Aspect Trajectory Similarity Measure}
\newacronym{lda}{LDA}{Latent Dirichlet Allocation}
\newacronym{lsi}{LSI}{Latent Semantic Indexing}
\newacronym{plsi}{PLSI}{Probabilitistic Latent Semantic Indexing}
\newacronym{mallet}{MALLET}{MAchine Learning for LanguagE Toolkit}
\newacronym{tmt}{TMT}{Stanford Topic Modeling Toolbox}
\newacronym{poi}{POI}{Point of Interest}
\newacronym{api}{API}{Application Programming Interface}
\newacronym{lbs}{LBS}{Location-based services}
\newacronym{aoi}{AOI}{Areas of Interest}
\newacronym{dbscan}{DBSCAN}{Density-Based Spatial Clustering of Applications with Noise}
\newacronym{hdbscan}{HDBSCAN}{Hierarchical Density-Based Spatial Clustering of Applications with Noise}
\newacronym{tfidf}{TF-IDF}{Term Frequency-Inverse Document Frequency}
\newacronym{prefixspan}{PrefixSpan}{\textbf{Prefix}-projected \textbf{S}equential \textbf{p}atter\textbf{n} mining}
\newacronym{gps}{GPS}{Global Positioning System}
\newacronym{kde}{KDE}{Kernel Density Estimation}


\begin{document}
\maketitle

% \chapter{Abstract}

\pagenumbering{roman}

\tableofcontents
\newpage

\listoffigures
\newpage

\listoftables
\newpage

\printglossary[type=\acronymtype]
\newpage

\pagenumbering{arabic}

\pagestyle{fancy}

 % ============================================ Introduction ============================================
\section{Introduction}
\subsection{Motivation}
How are cities distinguished from each other? The physical properties, like landmarks, road networks, and city structures, make the city distinctive. For instance, speaking of London, it is easy for people to come up with the London Eye, the Tower of London, Big Ben, etc. The city, however, is not only constituted by its physical properties, it is also a large human settlement \citep{goodall_penguin_1987,kuper_social_2013}. In his book “The Image of the City”, Kevin Lynch proposed the concept of the imageability of the city and discussed how the mental image is related to the physical qualities of the city \citep{lynch_image_1960}. According to the city perception survey  \underline{(Institute for Urban Strategies, 2020)}, the most frequent words used to describe London are Expensive, History, Big Ben, Culture, and Rain. People tend to describe the city based on what they see and experience, and how they feel about it rather than merely listing famous attractions. To better understand the city, it is far from enough to know only its physical properties. People interact with these physical properties and it is their mental images generated during the interaction that makes the city distinguished from others. Building the city perception map can enhance the city's characteristics, and helps to discover its uniqueness.

Social media data has become an increasingly popular source for discovering the city, as users can post spatially and temporally referenced information on platforms such as Twitter \footnote{\url{https://twitter.com/home}}, Flickr \footnote{\url{https://www.flickr.com/}}, and Foursquare \footnote{\url{https://foursquare.com/}}. The large number of Foursquare check-ins generated by users, for example, can be used to investigate how people move around the city \citep{ferreira_beyond_2015}, providing valuable insights into urban mobility patterns. Additionally, \acrfull{ugc} on social media platforms, like images, reviews, and \acrfull{poi}, offers the potential to extract city perception in a bottom-up approach, as people share their experiences and observations of different cities. Therefore, social media data can provide a precise and rich source of information for discovering and understanding cities.

The city perception is not static, it varies both spatially and temporally, which can even differ among different population groups. To reflect the dynamic nature of city perception, one can investigate people's movements. Spatially and temporally referenced social media data can enhance the study on human mobility \citep{beiro_predicting_2016}, particularly in constructing trajectories that provide semantic information about people's visiting purposes and impressions of the city. This extracted city perception is more akin to a city image that reflects the distinctiveness of the city, making it more attractive to people and resources. In an urban context, this perception can supplement public surveys to understand citizens’ needs and preferences. The abundance of social media data makes it possible to divide users into different population groups, and their city perceptions help to create a more livable city for people of diverse ages and socio-economic backgrounds. Thus, social media data offers a powerful tool for improving the quality of life for urban residents.

\subsection{Research Questions}
The city perception is typically collected through surveys involving a large number of participants, which can be labor-intensive and time-consuming. With the emergence of social media data, many studies have turned to \acrshort{ugc} to gain insight into how users perceive a city \citep{cranshaw_livehoods_2021,huang_user_2022}. However, while most studies put focus on specific regions of the city, little research has been conducted on city perception based on trajectories. Identifying meaningful places is a prerequisite for constructing trajectories. A place should not be merely a random point, but rather a space where people interact, with attributes based on human consensus. For instance, an area with grass and trees may not necessarily be considered a place, but if it attracts people and offers functionality for leisure purposes, it becomes a place. Using social media data, places can be identified based on users’ frequently visited locations. The Foursquare check-ins are often used to identify popular landmarks in a city \citep{ferreira_beyond_2015,ferreira_uncovering_2020,santos_uncovering_2018}. Foursquare venue names and categories enrich check-ins with attributes, making them more like places. In addition to these objective attributes, subjective attributes are also worth exploring. Flickr allows users to add tags to their photos, which can be a valuable source of data to enrich place attributes.

The trajectory is the chronological representation of people’s movements. Efforts have been made to extract movement patterns from trajectories to reveal the underlying visiting preferences \citep{vu_discovering_2019}. However, the trajectory should not be limited to geometric movements, the semantic information underlying trajectories is also valuable for exploration. When the semantic information of a trajectory is combined with its spatial and temporal data, it is referred to as a semantic trajectory \citep{yan_semantic_2011}. To understand visitor behaviors, most studies on semantic trajectories annotate the trajectories with attributes such as time, weekday, weather, etc. \citep{cai_mining_2018,petry_towards_2019}. However, in existing studies, the enrichment of semantic information for trajectories through place attributes has not been adequately considered. Semantic trajectories can vary across different groups of people, with locals and tourists organizing their trips differently based on local knowledge and online travel reviews. Moreover, different time spans can also result in different semantic trajectories, as people tend to have different visiting behaviors on weekdays and weekends. While existing studies have primarily focused on the visiting behaviors of locals and tourists \citep{domenech_using_2020,straumann_towards_2014}, the city perception of locals and tourists in different scenarios, like different time spans, is still under investigation.

To bridge the gap, this study aims to construct the semantic trajectories of locals and tourists in Greater London with Foursquare check-ins and Flickr tags. Greater London is an English-speaking city that attracts numerous tourists every year. Moreover, there are lots of Foursquare and Flickr users sharing check-ins and photos in Greater London, which lays the foundation for semantic trajectory construction. This study examines two research questions:

\textbf{RQ1: Which areas are more popular among locals and tourists at different time spans?}

Given the varying functionalities of different areas, they tend to attract diverse populations with specific visiting objectives. To investigate the distribution of areas that cater to distinct population groups, it is imperative to assess the degree of mixture between locals and tourists in these places \citep{li_analyzing_2018}. This evaluation can identify areas that mainly appeal to locals or tourists. By combining this evaluation with local knowledge, the visiting objectives of locals and tourists can also be revealed.

\textbf{RQ2: How do locals and tourists perceive the city along their semantic trajectories at different time spans?}

The perception of a city is subjective and can vary among both locals and tourists, with changes over time. To investigate city perception more accurately, constructing semantic trajectories that consider both population groups and time spans is useful. Specifically, semantic trajectories of locals and tourists during different times of day and week, including daytime and nighttime, as well as weekdays and weekends, will be constructed. By incorporating the semantic attributes of places into these trajectories, the city perception of both locals and tourists at various time spans can be interpreted and compared.

\clearpage

% ============================================ ||| Related Work||| ============================================
\section{Related Work}
% ============================= || City Perception || =============================
\subsection{City Perception}
% ====================== | UGC in City Perception | ======================
\subsubsection{UGC in City Perception}
% ============== specific perception ==============
City perception, which refers to how people experience and interact with the urban environment, has significant implications for the city's vitality and is a critical topic in the field of urban planning and design \citep{jacobs_death_1961}. It involves the study of individuals' subjective impressions, cognitive maps, and emotional responses towards urban spaces \citep{lynch_image_1960}. The large volume of \acrshort{ugc} available online makes data collection cost-effective, and it has proven to be a valuable data source for urban analytics, including the investigation of public perceptions towards cities. City perception research encompasses studies that aim to improve specific subjects within the city, as well as those that focus on understanding general city perception. Among studies that focus on specific subjects, landscape amenities have received widespread attention. For example, \cite{huang_user_2022} utilized Google Maps reviews to evaluate park performance and user experience, which showed the potential of using these reviews to enhance urban landscapes. There are also some studies investigating the soundscape of parks, as city perception can be reflected from an acoustic perspective. Such studies mainly focus on the evaluation of acoustic comfort and people's acceptability of the urban environment \citep{tse_perception_2012, liu_effects_2014}. Urban safety is another popular topic in city perception research. Some cities, despite being popular tourist destinations, suffer from natural disasters or negative publicity about crime, making perceived danger a worthy topic of investigation. \cite{yao_towards_2020} applied Tweets to build a real-time urban analytical and geo-visual system to provide early alerts for crises and emergencies. \cite{yang_crimetelescope_2018} collected crime data and Tweets to predict and visualize crime hotspots.

% ============== tourist interest - grid and AOI scale ==============
In studies aimed to investigate general city perceptions, analyses are carried out to identify popular areas at various spatial scales. A grid-based approach has been employed to detect tourist attractions. The study area is divided into equal-sized grids, and the concentration of \acrshort{ugc} within grids is measured. Spatial autocorrelation indices, such as Moran's I and Getis-Ord G statistics, were usually used to identify spatial clusters of tourism activities \citep{garcia-palomares_identification_2015, kim_coastal_2021}. Some studies go beyond exploring the distribution of tourist hotspots and extract semantic information from these hotspots. For example, \cite{li_analyzing_2018} created the location-based word-cloud maps with Flickr tags to better identify the exact attractions in each cluster, enriching aspects of city perception. \cite{bahrehdar_description_2018} delved deeper into Flickr tags by generating semantic topics with topic modeling, and labeled grids with meaningful topics to map users' perception of the space. A non-grid-based approach can also be utilized to discover the spatial patterns of tourist attractions. For instance, Tweets, Flickr images, and Foursquare check-ins were used to detect \acrfull{aoi} with clustering techniques such as K-Means clustering \citep{hartigan_algorithm_1979}, \acrfull{dbscan} \citep{ester_density-based_1996}, and self-developed algorithms \citep{hu_extracting_2015, hasnat_identifying_2018, cranshaw_livehoods_2021}. There are also studies extracting semantic information associated with \acrshort{aoi}. \cite{dunkel_visualizing_2015} clustered Flickr images and subsequently mapped the tags associated with each cluster, with the size reflecting the frequency, which presented a comprehensive overview of people's perceptions of the study areas. \cite{zhou_detecting_2015} applied \acrshort{dbscan} to detect Flickr images communities and then employed random forest to classify Flickr tags into three categories based on the spatial, temporal, and user features. This helped describe detected communities with more precise word clouds. \cite{jailani_machine_2021} used \acrfull{tfidf} to assign weights to keywords of Flickr data, including tags, titles, and descriptions, and utilized \acrshort{dbscan} to cluster weighted keywords, thus the discovered \acrshort{aoi} were integrated with intrinsic semantic information. \cite{santos_uncovering_2018} collected reviews about places from Google Places and Foursquare tips, and generated perception maps to uncover how the urban outdoor areas were expressed in social media.

% ============== tourist interest - trajectory scale ==============
Constructing trajectories serves as an alternative approach to discovering how visitors perceive a city, as it reflects their movement patterns. \cite{girardin_digital_2008} utilized people's mobile phone calls and Flickr images to investigate the visitor flows among major visitor attractions in Rome, Italy. They applied the Origin-Destination matrix to understand the preferences of visitors. Some studies use the trajectory network to detect the movement patterns of visitors. The weighted network graph was constructed from the clusters of Flickr and Twitter data, and then the network analysis like betweenness and eigenvector centrality was performed to extract popular attractions and routes \citep{straumann_towards_2014, hu_graph-based_2019}. Some studies construct trajectories based on the street layout. For example, \cite{mohino_identifying_2018} identified the main tourist routes of Flickr users along the street network, and \cite{domenech_using_2020} established the hierarchy of the street network based on the number of trajectories passing through. This helped to better understand the city structure and context. \cite{yin_diversified_2011} also ranked the street-based trajectories with various ranking methods, which contributed to the location recommendation at the trajectory level. Researchers have attempted to construct semantic trajectories to interpret the movement patterns more meaningfully. Different from raw trajectories that only contain spatial and temporal information, semantic trajectories are also annotated with higher-level semantic information at each point. \cite{wan_semantic-geographic_2017} took the venue categories of Sina check-ins into consideration when building users' semantic-graphic traces, and detected their movement patterns with a density-based clustering algorithm. In order to gather further insights into visitors' characteristics and activity preferences, topic modeling was used to analyze the venue categories of their check-ins \citep{vu_discovering_2019, ferreira_uncovering_2020}. In addition to the venue category, the weather and time were also considered in the construction of semantic trajectories \cite{cai_mining_2018,liu_stccd_2020}.

% ====================== | Topic Modeling of UGC | ======================
\subsubsection{Topic Modeling of UGC}
\acrfull{nlp} has emerged as a powerful tool for information retrieval. Within \acrshort{nlp}, topic modeling has become increasingly popular in recent years, especially in text mining for \acrshort{ugc}. Topic modeling branches off from the area of generative probabilistic modeling and is used to identify latent themes within a large corpus. Given a set of documents,  topic modeling allows researchers to gain insights into topics being discussed by people \citep{sui_inferring_2013}. In topic modeling, a \textit{word} or \textit{term} is a single token, which is the fundamental unit of individual data. A \textit{document} refers to a piece of text comprising multiple words. A \textit{corpus} is a collection of documents and serves as the basis for topic modeling. A \textit{vocabulary} represents all distinct words in a corpus. A \textit{topic} is a latent theme discovered in topic modeling, and it is characterized as a probability distribution spanning a given vocabulary \citep{vayansky_review_2020}. Figure~\ref{fig:topic_modeling} shows the process of topic modeling. With the corpus as the input, topic modeling is performed to generate a set of topics, each of which represents a cluster of words with the probability of belonging to the topic. Furthermore, the results of topic modeling include the distribution of topics in each document, which indicates the degree of association between the topics and the document, as well as the frequency of words within each topic. The origin of the topic model is \acrfull{lsi} proposed by \cite{papadimitriou_latent_2000}, but \acrshort{lsi} is not a probabilistic model. To address this limitation, \cite{hofmann_unsupervised_2001} introduced \acrfull{plsi} and subsequently, \cite{blei_latent_2003} proposed \acrfull{lda}, which is a more complete generative probabilistic model \citep{liu_overview_2016}. Compared with \acrshort{plsi}, \acrshort{lda} generates better disambiguation of words and more precise identification of topics in documents due to its consideration of a sparse Dirichlet prior in the topic distribution \citep{barde_overview_2017}.

The \acrshort{lda} method has demonstrated its usefulness as a valuable tool for understanding unstructured data. In practice, several tools are available for implementing the \acrshort{lda} model. \acrfull{mallet} \footnote{\url{https://mimno.github.io/Mallet/index}} is a Java-based package that provides efficient and sampling-based implementations of \acrshort{lda} related models. Gensim \footnote{\url{https://radimrehurek.com/gensim/}} is a Python library that implements popular topic modeling algorithms, including \acrshort{lda} and \acrshort{lsi}. \acrfull{tmt} \footnote{\url{https://downloads.cs.stanford.edu/nlp/software/tmt/tmt-0.4/}} is another tool that trains topic models to create summaries of the text. An evaluation of \acrshort{mallet} and Gensim was conducted to compare their performances \citep{ebeid_mallet_2016}. To facilitate the interpretation of \acrshort{lda} results, \cite{sievert_ldavis_2014} developed LDAvis \footnote{\url{https://github.com/cpsievert/LDAvis}}, a web-based interactive visualization of topics. LDAvis allows users to select a topic to reveal the most relevant terms for that topic. Users can also select a term to reveal its conditional distribution over topics (Figure~\ref{fig:ldavis}). LDAvis leverages the R language, specifically the shiny package, to enable users to visualize topics. Additionally, a Python library, pyLDAvis \footnote{\url{https://github.com/bmabey/pyLDAvis}} was developed to provide interactive topic model visualization in Python, which is a port of the R LDAvis package.

Topic modeling using \acrshort{lda} has been widely adopted for extracting information from \acrshort{ugc}. For instance, Flickr tags have been utilized for investigating the properties of places and semantic similarity of streets \citep{bahrehdar_description_2018, bahrehdar_streets_2020}. In addition, the venue categories of Foursquare check-ins have been used to extract semantic information about places. The \acrshort{lda} has been employed to the venue categories of users' check-ins along their movements to depict different types of users with distinct character profiles \citep{ferreira_uncovering_2020} and discover implicit activity preferences of users \citep{vu_discovering_2019}. Furthermore, Google reviews have been found to be a good data source to extract leisure activity potentials in urban space \citep{van_weerdenburg_where_2019}.


\begin{figure}[!h]
\centering
\includegraphics[width=0.7\textwidth]{figures/topic_modeling.png}
\caption{\label{fig:topic_modeling}Framework of topic modeling \citep{usmani_natural_2021}.}
\end{figure}

\begin{figure}[!h]
\centering
\includegraphics[width=1\textwidth]{figures/ldavis.png}
\caption{\label{fig:ldavis}The layout of LDAvis \citep{sievert_ldavis_2014}.}
\end{figure}

% ====================== | Identification of Locals and Tourists | ======================
\subsubsection{Identification of Locals and Tourists}
Distinguishing between locals and tourists is crucial when investigating how different people perceive a city. Locals and tourists might have various preferences and behaviors while visiting a city, and analyzing their patterns separately can yield valuable insights. The time interval is a commonly used indicator for identifying locals and tourists. Typically, locals are assumed to stay in the city for longer periods than tourists. Researchers have used the time interval between a user's first and last posts within a city boundary on social media platforms to estimate the length of his stay, with time intervals ranging from 10 days to 30 days \citep{girardin_digital_2008, hu_graph-based_2019, hopken_flickr_2020}. For example, given a 30-day threshold, a Twitter user whose first and last Tweets were posted within a 20-day time interval would be classified as a tourist. However, it is important to note that some locals may share content only for short periods, and some tourists may stay longer and continue sharing content. Another indicator of determining locals and tourists is user profiles. Social media platforms like Foursquare, Flickr, and Twitter, offer \acrshort{api} to collect user profiles containing information on their city of residence and hometown. Previous studies have demonstrated the feasibility of using these profiles to identify users' origins \citep{ferreira_beyond_2015, li_analyzing_2018}. However, this approach has some limitations, as many users may leave their profiles incomplete without providing the city of residence. \cite{ferreira_uncovering_2020} combined both the time interval and user profiles to overcome these limitations. If a user's time interval suggests that he is a local, but his profile indicates that he is from another city, he would be classified as a tourist. There are also other indicators for the identification of locals and tourists. In addition to the time interval, \cite{hallot_who_2015} also analyzed the categories of the place visited and the frequency of visits within the same place to extract tourists. \cite{yang_identifying_2021} assumed that locals and tourists might have different numbers of Foursquare check-ins, total travel distances, and time intervals, and applied K-means clustering to separate users into locals and tourists based on the above indicators. \cite{hasnat_identifying_2018} used the number of Tweets within the geographical boundary as an indicator to identify tourists. For example, if a user posted fewer Tweets within the geographical boundary of Florida between 12 am and 6 am than outside the boundary, he would be assumed as a local. Overall, these approaches help researchers to effectively differentiate between locals and tourists, enabling them to investigate the visiting behaviors of different groups of people.


% ============================= || Place || =============================
\subsection{Place}
% ====================== | Place Conceptualization | ======================
\subsubsection{Place Conceptualization}
\underline{definition of place, which will be used to explain why places are clusters of check-ins instead of merely check-ins}
\underline{add place studies mentioned in colloquium}

City perception can be shaped by individuals' perceptions of various places. To explore place-based perception, the first step is to conceptualize the place. According to \cite{tuan_place_1975}, place is created by human beings for human purposes, and it encompasses not only geometrical and ideographic perspectives but also an experiential perspective. It is important to distinguish between space and place. Space is defined as a continuous and unrestricted area that can be free to use or occupied, moreover, it is abstract and lacks content. While place is a segment of geographical space loaded with human meaning, offering the potential for human interaction \citep{tuan_place_1975, agnew_space_2011, cresswell_place_2014}. \cite{relph_place_1976} developed an \textit{insideness} scale to illustrate the social relationships of a place, which includes the knowledge of the physical details of place, sense of connection with a community, and a personal connection with place. Moreover, \cite{williams_measurement_2003} employed the place attachment \citep{altman_place_1992} as the scale to identify and measure the meanings of places based on the place identity \citep{proshansky_city_1978,proshansky_place_1983} and place dependence \citep{stokols_people_1981}. As the definition of place has evolved, researchers have contributed to the conceptualization of place dimensions. \cite{jorgensen_comparative_2006} described place with three dimensions: (1) place-specific beliefs (\textit{place identity}), (2) emotions (\textit{place attachment}), and (3) behavioral commitments \textit{place dependence}. \cite{agnew_space_2011} conceptualized the place with three dimensions: (1) \textit{location}, which refers to the physical position of a place represented by its name and coordinates, (2) \textit{locale}, which encompasses the properties and affordance of a place, and (3) \textit{sense of place}, which is associated with the sentiments and emotions of individuals who visit the place \citep{bahrehdar_description_2018}. It is noteworthy that according to the affordance theory, \textit{affordance} can shape behavior and guide actions of individuals \citep{gibson_theory_1977}, thus the affordance of place can influence how people perceive it. 

% ====================== | Place Dimensions Representation | ======================
The dimension \textit{location} is typically represented by the toponym. For the dimension \textit{locale}, 
\cite{wartmann_characterizing_2016,wartmann_describing_2018} have refined it with categories of landscape elements, indicating the possibility of representing this dimension with place categories. \cite{koirala_social_2015} categorized places based on tourism ontologies, which include (1) leisure, (2) restaurant, (3) attraction, (4) emergency service, (5) transport, (6) accommodation, and (7) other buildings. \cite{liu_stccd_2020} constructed a location category hierarchy tree based on daily purposes, including (1) work/study, (2) food, (3) entertainment, (4) traffic, and (5) live. \acrfull{lbs} providers also categorize places to offer place-related \acrshort{api} for users to search for \acrshort{poi} by category (Table~\ref{tab:lbs_place_categories}). However, these place categories are typically divided based on specific services provided by \acrshort{lbs} providers, and the categorization bias might lead to incomplete categories. For instance, Waze provides services for driving directions and live traffic conditions updates, which means that its place categories tend to be traffic-oriented. Tripadvisor is an online travel agency that provides guidelines for visitors, and it divides places mainly based on where to stay and what to do during a trip. To address this limitation, some studies have modified the place categories provided in \acrshort{lbs} platforms based on their research objectives. For example, in the case of Foursquare check-ins, the place category system is structured hierarchically and there are subcategories under categories. \cite{ferreira_uncovering_2020,yang_identifying_2021} improved the place categories by moving some subcategories into other categories or new categories to gain a better understanding of visitors' behaviors. The dimension \textit{sense of place} is closely linked to people's emotional attachment to the place. Conducting surveys is a way to gather information on people's emotional relationships with places and to understand their positive or negative experiences \citep{manzo_for_2005}. However, surveys can be time-consuming, and \acrshort{ugc} can provide an alternative means of exploring people's relationships with places. \cite{wang_using_2015} utilized Foursquare check-ins to evaluate the performance of four different clustering algorithms in identifying meaningful places. \cite{sui_inferring_2013} applied \acrshort{lda} to georeferenced travel blogs to generate meaningful topics describing places. \cite{hallot_who_2015} combined Google Place reviews and Foursquare check-ins to retrieve place-based semantics, enabling the inference of additional information about users based on their movements.

\begin{table}[!h]
\centering
\caption{\label{tab:lbs_place_categories}Overview of place categories in various LBS providers.}
\begin{adjustbox}{max width=\textwidth, margin=0cm}
\begin{threeparttable}
\begin{tabular}{lp{10cm}l} \hline
LBS Provider & Place Categories & Related API\\ \hline
Google Maps\tnote{a} & Airport, Amusement Park, Bank, Cafe, Embassy, Gym, Hospital, Library, Museum, Zoo, etc. (Google Maps divides places into 96 categories in total ) & Places API \\
Esri ArcGIS\tnote{b} & Arts and Entertainment, Education, Food, Land Features, Nightlife Spot, Parks and Outdoors, Professional and Other Places, Residence, Shops and Service, Travel and Transport, Water Features & Geocoding REST API \\
Foursquare\tnote{c} & Arts and Entertainment, Business and Professional Services, Community and Government, Dining and Drinking, Event, Health and Medicine, Landmarks and Outdoors, Retail, Sports and Recreation, Travel and Transportation & Places API \\
Tripadvisor\tnote{d} & Hotel, Restaurant, Attraction & Location Search API \\
Waze\tnote{e} & Parking Lot, Car Services, Transportation, Professional and Public, Shopping and Services, Food and Drink, Culture and Entertainment, Other, Lodging, Outdoors, Natural Features & Waze API \\
HERE\tnote{f} & Eat and Drink, Going Out-Entertainment, Sights and Museums, Natural and Geographical, Transport, Accommodations, Leisure and Outdoor, Shopping, Business and Services, Facilities, Areas and Buildings & Geocoding \& Search API \\
TomTom\tnote{g} & Agriculture, Beach, Castle, Factory, Garden, School, Railroad Stop, Temple, etc. (TomTom divides places into 778 categories in total ) & Category Search API \\ \hline
\end{tabular}
\footnotesize{Source: LBS provider websites as of May 2023.}
\begin{tablenotes}\footnotesize
\item[a] \url{https://www.google.com/maps}
\item[b] \url{https://www.arcgis.com/index.html}
\item[c] \url{https://foursquare.com/}
\item[d] \url{https://www.tripadvisor.com/}
\item[e] \url{https://www.waze.com/live-map/}
\item[f] \url{https://www.here.com/}
\item[g] \url{https://www.tomtom.com/}
\end{tablenotes}
\end{threeparttable}
\end{adjustbox}
\end{table}

 
% ====================== | Place Popularity Measurement Among Locals and Tourists| ======================
\subsubsection{Place Popularity Measurement Among Locals and Tourists}
The distribution of locals and tourists can provide valuable insights into the popularity of places among two groups of people. In previous studies, researchers have employed various indices to measure the patterns of population distributions. One commonly used index is the \textit{index of dissimilarity}, which was proposed by \cite{sakoda_generalized_1981} to measure the distribution of two populations across geographic areas. \cite{li_analyzing_2018} applied this index to represent the degree of mix between locals and tourists to study their spatial interactions. The \textit{index of dissimilarity} is calculated as the sum of the difference ratio of locals and tourists in each place, reflecting the overall distribution of locals and tourists over the study area. However, the difference ratio itself is also a good indicator to measure the popularity of individual places. In addition, \cite{mcelroy_applying_1993} formulated the \textit{density ratio} (the number of visitors multiplied by the average length of stay, divided by land area multiplied by 365) and the \textit{penetration ratio} (the number of visitors multiplied by the average length of stay, divided by the population multiplied by 365) to reveal the degree of tourists influx into an area. These two indices were typically used to measure the tourist carrying capacity \citep{mcelroy_applying_1993,thomas_tourist_2005}. However, the \textit{penetration ratio}, which shows the proportion of tourists in a specific place at a temporal scale, can also be used to measure the popularity of places among two groups of people. \cite{mcelroy_applying_1993,mcelroy_tourism_1998} developed the \textit{tourism penetration index} to measure the degree of tourism penetration. The calculation of \textit{tourism penetration index} involves three variables: (1) per capita visitor spending, (2) daily visitor densities per 1,000 population, and (3) hotel rooms per square kilometer, as indices to measure the tourism impact. The \textit{tourism penetration index} is then calculated as the unweighted average of the three standardized indices. In the context of place popularity measurement, the second variable, which reveals the average tourist density, also indicates the place popularity among locals and tourists, and standardizing this index helps to investigate the distribution of these two groups of people for each place. Furthermore, the \textit{tourist ratio}, calculated as the ratio of the number of tourists to the number of residents in a specific area, is also an indication of the intensity of tourist influx \citep{faulkner_framework_1997}. 


% ============================= || Semantic Trajectory || =============================
\subsection{Semantic Trajectory} \label{semantic_trajectory}

% ====================== | Semantic Trajectory Construction | ======================
\subsubsection{Trajectory Construction}
% ============== introduction to the semantic trajectory ==============
Detecting the movement patterns of visitors from their trajectories helps to understand their visiting behaviors. However, interpreting these patterns can sometimes be challenging due to the lack of contextual information. To address this, attempts have been made to integrate semantic descriptions into raw trajectories to build semantic trajectories. A trajectory can be assumed as a sequence of stops and moves \citep{spaccapietra_conceptual_2008}. Stops represent specific points along the trajectory where the moving object stays for a certain duration, typically denoting locations of interest. For instance, in tourism studies, stops could include sightseeing spots, hotels, airports, etc. \citep{yuan_review_2017}. Moves represent the transitions between stops in a trajectory, indicating the segments from one stop to the next. Unlike raw trajectories that only capture spatial or spatiotemporal properties (Figure~\ref{fig:raw_trajectory_example}), semantic trajectories enrich stops and moves with contextual data such as weather, transport means, and place type (Figure~\ref{fig:semantic_trajectory_example}). The integration of semantic information enables the extraction of meaningful trajectory behaviors. For example, in the context of tourism, a tourist behavior can be identified if the trajectory begins and ends at an accommodation place, with several stops at museums or tourist attractions \citep{parent_semantic_2013}. Constructing a semantic trajectory involves identifying stops, which are the important places of trajectory, based on the trajectory data. Subsequently, stops or moves are annotated with semantic information, and this process is also called semantic enrichment.

\begin{figure}[!h]

\centering
\begin{subfigure}{0.5\textheight}
\centering
\includegraphics[width=0.9\linewidth]{figures/raw_trajectory_example.png} 
\caption{Raw trajectory.}
\label{fig:raw_trajectory_example}
\end{subfigure}
\begin{subfigure}{0.5\textheight}
\centering
\includegraphics[width=0.9\linewidth]{figures/semantic_trajectory_example.png}
\caption{Semantic trajectory}
\label{fig:semantic_trajectory_example}
\end{subfigure}

\caption{Example of trajectories \citep{ferrero_mastermovelets_2020}.}
\label{fig:trajectory_example}
\end{figure}



% ============== GPS data ==============
\acrfull{gps} data is widely used for constructing trajectories. Various approaches exist for identifying stops from \acrshort{gps} data. One method involves detecting stops by examining the absence of \acrshort{gps} signals or when the velocity remains zero during a specific time interval \citep{ashbrook_using_2003}. However, the presence of signal errors decreases the accuracy of identifying actual stops using this technique. Alternatively, other approaches consider both the \acrshort{gps} data and background geographic information to identify stops. \cite{alvares_model_2007} developed an algorithm named SMoT (Stops and Moves of Trajectories) for extracting stops and moves. In their study, a stop is defined as a location where an object remains for a minimal amount of time, and it is annotated with the corresponding place type and timestamp. Density-based clustering algorithms are also employed to identify stops. \cite{palma_clustering-based_2008} treated stops as \acrshort{poi} and utilized various \acrshort{dbscan} algorithms to cluster points as places based on the speed between two points. To enrich trajectories with semantic enrichment, different techniques have been employed, such as semantic regions, semantic lines, and semantic points, to annotate trajectories \citep{yan_semantic_2013}. Semantic regions refer to meaningful geographic areas, such as land use, and stops in trajectories can be annotated with semantic information by considering their topological correlation with these regions. Semantic lines typically represent transport networks, the movements within trajectories can be linked to road segments through map-matching algorithms, helping to infer the transportation mode, such as walking, driving, or using public transportation like metros. Semantic points correspond to \acrshort{poi} with meaningful place types, like restaurants, museums, and train stations. Stops in trajectories can be linked to the closest \acrshort{poi} to annotate semantic information. Furthermore, stops and moves can be annotated with activities, like staying at home, having lunch at a restaurant, or working at a company, by considering the time individuals spend at different \acrshort{poi} and their routines \cite{parent_semantic_2013}.

% ============== UGC data ==============
In addition to utilizing \acrshort{gps} data, the growing utilization of \acrshort{ugc} data shows its potential in semantic trajectory construction by providing more semantic information beyond the spatial location. Trajectories have been built using check-ins gathered from platforms like Foursquare, Gowalla, and Brightkite. In these cases, a stop is defined as a semantic \acrshort{poi} with a minimum of 10 check-ins, and consecutive check-ins within a 10-min threshold are removed as duplicates \citep{petry_towards_2019,ferrero_mastermovelets_2020}. Another approach employed by \cite{nin_tweets_2014} involved aggregating geotagged tweets to create trajectories, associating them with Foursquare categories to construct the Semantic Origin Destination Matrix of Foursquare categories. Geotagged photos from Flickr have also served as a valuable data source for building semantic trajectories. \cite{cai_mining_2018} conducted semantic trajectory clustering using geotagged photos from Flickr to mine mobility patterns. Additionally, they identified semantic regions of interest where a high density of trajectories passed, providing more meaningful descriptions of the trajectories. Similarly, \cite{yang_quantifying_2017} utilized geotagged Flickr photos to extract tourist trajectories, expanding the trajectory dimensions from topological and temporal spaces to semantic spaces. This allowed for a better understanding of travel motifs and the discovery of meaningful patterns of tourist behavior. In terms of semantic enrichment, various attributes have been employed to enrich the semantic information of trajectories, including place type, price tier, rating, day of the week, time of day, and weather information have been used to enrich semantic information for trajectories \cite{cai_mining_2018,petry_towards_2019,liu_stccd_2020,ferrero_mastermovelets_2020}. Place type is a fundamental contextual piece of information associated with semantic trajectories, which can be collected from sources such as \href{http://www.geonames.org/}{GeoNames} or represented by venue categories derived from check-ins. Studies that utilized check-ins to construct trajectories also considered the price tier and rating associated with the check-ins. These attributes, together with place type, provide additional information about the visited places. Day of the week and time of day assist in interpreting mobility patterns with greater time granularity. For instance, it is possible to identify typical trajectories where an individual commutes from home to work at 8 am, has lunch at a restaurant at 12 pm, and so on during weekdayss. The incorporation of weather information helps to capture the influence of weather conditions on individuals' movement patterns.

% ====================== | Semantic Trajectory Pattern Mining | ======================
\subsubsection{Semantic Trajectory Pattern Mining}
% ============== introduction + distance function + similarity measure ==============
The movement patterns of individuals can be mined from semantic trajectories to understand their behaviors. Two primary techniques employed for the discovery of movement patterns are clustering and classification \citep{parent_semantic_2013}. These techniques group trajectories that share similar information. The key distinction between clustering and classification is that clustering is an unsupervised learning technique, which does not rely on prior knowledge of target groups. On the other hand, classification is a supervised learning technique that requires prior knowledge to determine target groups before the learning progress. To cluster semantic trajectories, it is important to construct similarity matrix using appropriate distance measures. Various distance functions are available to measure diverse attributes. In the context of semantic trajectories, spatial distance is typically evaluated using either the Euclidean Distance function or the Haversine Distance function. The Euclidean Distance function calculates the straight-line distance between two points and is suitable for coordinates in a projected coordinate system. Conversely, the Haversine Distance function measures the great-circle distance between two points on the Earth's surface and it is suitable for latitude and longitude coordinates in a geographic coordinate system. Semantic trajectories may also encompass attributes with categorical values, such as place type and weather. For measuring the distance between such attributes, discrete distance functions like the Hamming Distance function and Jaccard Distance function are commonly employed. These functions are particularly useful for binary or categorical data, allowing for the quantification of the distance between different attribute values.

To extract typical trajectories, it is essential to measure trajectory similarity. Commonly used similarity measures for trajectories include Dynamic Time Warping (DTW) \citep{berndt_using_1994}, Multidimensional Dynamic Time Warping (MD-DTW) \citep{ten_holt_multi-dimensional_2007}, Longest Common SubSequence (LCSS) \citep{vlachos_discovering_2002}, Edit Distance (EDR) \citep{chen_robust_2005}. A comparative study of these trajectory similarity measures was conducted by \cite{tao_comparative_2021}. DTW measures the distance between time series and is specifically suitable for numerical values. It determines the optimal alignment of two sequences to identify the contiguous path with the minimum total distance between the series. However, DTW only considers one dimension. To address this limitation, MT-DTW was developed to adapt DTW for multidimensional sequences. It assigns weights to different dimensions and constructs the distance matrix for each pair of elements in the sequences by normalizing the distance values across all dimensions. Both DTW and MT-DTW have a drawback in that they are sensitive to noises, such as distant elements. LCSS, proposed as a similarity measure for raw trajectories, overcomes this limitation by introducing the distance and matching threshold to identify the longest common subsequence. If the distance between two points falls within the matching threshold, a similarity value of 1 is assigned; otherwise, a similarity value of 0 is assigned. LCSS can also be extended to measure trajectories with additional dimensions, such as semantics. EDR, derived from LCSS, also utilizes the matching threshold with binary values (0, 1) to represent the distance. It calculates the distance between trajectories by seeking the sequence with the minimum number of inserts, deletes, and replacements of points required to transform one trajectory into another. However, most trajectory similarity measures only consider a fixed number of dimensions and deal with fixed types of values. \cite{furtado_multidimensional_2016} proposed the Multidimensional Similarity Measure (MSM) to compute the similarity of trajectories with multiple dimensions, including space, time, and semantics. MSM assumes that different dimensions may have varying levels of importance in different problems, thus assigning weights to dimensions and utilizing thresholds to calculate similarity scores. This approach increases the flexibility of similarity measures. However, MSM does not consider the sequence of the movement, defining two trajectories as similar even if they visit the same place types in different orders. \cite{petry_towards_2019} developed the MUItiple-aspect TrAjectory Similarity (MUITAS) method to measure the similarity of trajectories with heterogeneous semantic dimensions. MUITAS supports different relationships among dimensions by introducing features. A feature represents a unit of analysis within a trajectory, and multiple aspects constitute the feature. Different weights can be assigned to different features based on the importance of aspects within the feature. This approach enhances the ability to capture the complexity of trajectory similarity.

% ============== semantic trajectory clustering ==============
Trajectories can be clustered using various methods, including density-based, hierarchical-based, spectral-based, and community-based trajectory clustering techniques \cite{liu_stccd_2020}. A density-based clustering method identifies clusters with high object density within a given area, enabling the detection of clusters with arbitrary shapes. One of the most renowned density-based clustering algorithms is \acrshort{dbscan}, which relies on two parameters, namely, \textit{Eps} (radius of the neighborhood) and \textit{MinPts} (minimum number of neighbors within the radius). Ordering Points To Identify the Clustering Structure (OPTICS) proposed by \cite{ankerst_optics_1999} is another density-based clustering method. OPTICS orders objects instead of clustering them as \acrshort{dbscan} does, and the ordered objects can be grouped into clusters based on the reachability distance. For instance, \cite{cai_mining_2018} applied OPTICS to cluster trajectories with multiple semantic dimensions constructed using geotagged Flickr photos, and identified semantically common trajectory patterns. In the hierarchical-based clustering approach, \cite{zhang_hierarchical_2018} proposed a hierarchical trajectory clustering method based periodic pattern mining, incorporating semantic spatiotemporal information. This method extends Traclus (Trajectory Clustering) \citep{lee_trajectory_2007} and builds upon HDBSCAN (Hierarchical Density-Based Spatial Clustering of Applications with Noise) \citep{campello_density-based_2013} to detect hierarchical clusters. Regarding community-based trajectory clustering, \cite{el_mahrsi_graph-based_2013} employed a modularity-based community detection algorithm to group frequently visited road segments from different trajectories. This approach enables the discovery of a hierarchy of nested clusters of road segments. \cite{liu_todmis_2013} proposed Trajectory cOmmunity Discovery using Multiple Information Sources (TODMIS) to mine communities from trajectories, which combines additional information, such as velocity and semantics, with raw trajectories and applied dense sub-graph detection to discover the set of distinct communities.


% ============== semantic trajectory classification ==============
Classification is a process that groups similar trajectories into predefined classes. In the context of semantic trajectories, \cite{lee_trajectory_2007} initially clustered interesting line segments derived from trajectories. Subsequently, they assigned a class to each cluster and classified other trajectories by placing them into these predefined classes. To delve into the analysis of trajectory data, \cite{giannotti_unveiling_2011} developed M-Atlas, a comprehensive system that facilitates the exploration of both raw and semantic clustering and classification, and discovered human mobility through querying and mining trajectory data. Within their study, trajectories were clustered, and new trajectories were classified by assigning them to the identified clusters. \cite{ferrero_mastermovelets_2020} introduced a new parameter-free method known as MasterMovelets. This approach identifies the most relevant sub-trajectories while considering various combinations of dimensions. By leveraging this method, trajectory classification can be achieved efficiently and effectively. 

% ============== sequential pattern mining ==============
Clustering and classification techniques are employed to group semantic trajectories that share similar attributes. In addition to these methods, sequential pattern mining serves as a complementary approach to extract frequently occurring mobility patterns within trajectories. Sequential pattern mining allows for the occurrence of frequent points in a specific temporal order, which means that each point can occur multiple times as long as it is visited several times during the same period. One of the pioneering algorithms in sequential pattern mining is the Generalized Sequential Patterns (GSP) algorithm proposed by \cite{srikant_mining_1996}. This algorithm accommodates the specification of (1) time constraints between adjacent elements of the sequential pattern, (2) sliding time windows of the transaction, and (3) user-defined taxonomies. In the study conducted by \cite{hopken_flickr_2020}, the GSP algorithm was applied to identify behavioral patterns of tourists using Flickr data. The authors compared the typical trajectories mined through association rule analysis and sequential pattern mining. \cite{pei_mining_2004} introduced a \acrfull{prefixspan} algorithm to mine sequential patterns through a pattern-growth approach, which consumes less memory space than GSP. Furthermore, \cite{yin_diversified_2011} applied the \acrshort{prefixspan} method to discover the trajectory patterns of Flickr users. They also employed various ranking methods to find top-ranked patterns based on different criteria.

% ============================= || Research Gaps || =============================
\subsection{Research Gaps}
City perception can be investigated at different scales, such as grid-based or \acrshort{aoi}-based, but there is limited research exploring how the perception of a city can be understood through trajectories. Previous studies have demonstrated the feasibility of constructing trajectories with \acrshort{ugc} data. While geotagged social media data like Tweets, Flickr photos, and Foursquare check-ins, have been utilized to build trajectories, there is a lack of research that integrates \acrshort{ugc} data as semantic information for trajectory construction and the study of city perception. Most studies directly assign information extracted from \acrshort{ugc} data, such as place type, rating, and price tier, to trajectories as the semantic dimension. However, \acrshort{ugc} also contains latent information which requires additional techniques for extraction. This study assumes that a place can be characterized by three dimensions: location, locale, and sense of place. It integrates these dimensions into the construction of semantic trajectories and enriches them through direct information assignment and topic modeling. For the construction of semantic trajectories, existing approaches are not fully suitable for exploring perception from a place-based perspective. Additionally, city perception varies among different groups of people and over different time spans, making it worthwhile to analyze more precise city perceptions comparatively. To capture the semantic information in trajectories, this study leverages \acrshort{ugc} data and integrates the place dimensions, aiming to identify the movement patterns of both locals and tourists over different time spans. By incorporating \acrshort{ugc} data, this study seeks to provide insights into city perception from a holistic and nuanced perspective.

\clearpage

% ============================================ ||| Study Area and Data Preprocessing ||| ============================================
\section{Study Area and Data Preprocessing}
\subsection{Study Area}
The study area is located in Greater London in Great Britain (Figure~\ref{fig:study_area}). As an English-speaking
city, Greater London covers an area of 1,572 \(km^2\) and has a population of 9.5 million inhabitants. According to tourism statistics of City of London, before the Covid-19 pandemic, Greater London attracted approximately 21 million visitors annually from across the world. Of these, 19.7 million visitors were day trippers, while 1.3 million visitors stayed overnight. The high volume of visitors makes Greater London an ideal location for studying the city perception of both locals and tourists. The boundary of Greater London used in this study was obtained from LONDON DATASTORE website \footnote{\url{https://data.london.gov.uk/dataset/statistical-gis-boundary-files-london}}.

\begin{figure}[h!]
\centering
\includegraphics[width=1\textwidth]{figures/study_area.png}
\caption{\label{fig:study_area}Study area.}
\end{figure}

\subsection{Data Preprocssing} \label{data_preprocessing}
Foursquare data and Flickr data in Greater London from April 3, 2012 to September 16, 2013 were collected to investigate how people move around the city and how they perceive it.

This study examines the city perceptions of locals and tourists, during the daytime and nighttime, as well as on weekdays and weekends. Therefore, it is crucial to differentiate between these two population groups across different time spans. To distinguish between locals and tourists, a combination of user profiles and time intervals was utilized for identification. The user profile served as the primary criterion, with individuals whose hometown or city of residence listed as London in their profiles categorized as locals. In cases where the user profile was unavailable, the time interval between the user's initial and final Foursquare or Flickr posts in Greater London was employed. Users with a time interval exceeding 30 days were classified as locals. Regarding time spans, daytime is defined as 6 am to 6 pm, while the remaining hours are considered nighttime. Weekdays encompass Monday to Friday, while the weekend comprises Saturday and Sunday.

\subsubsection{Foursquare Data}
Foursquare is a platform for users to check in at venues and share their experiences through reviews. As of 2022, Foursquare has more than 55 million monthly active users worldwide. The Foursquare data used in this study was obtained from the Global-scale Check-in Dataset \cite{yang_nationtelescope_2015}, containing two datasets: (1) Foursquare check-ins and (2) Foursquare \acrshort{poi} from April 3, 2012 to September 16, 2013, on a global scale.

For the dataset Foursquare check-ins, a total of 187,336 check-ins shared by 9,717 users in Greater London were extracted. This dataset includes the following fields:
\begin{itemize}
    \item User ID
    \item Venue ID
    \item UTC time
    \item Timezone offset
\end{itemize}

Another dataset Foursquare \acrshort{poi} stored Foursquare venues, and there were 27,608 \acrshort{poi} in Greater London. This dataset contains the following fields:
\begin{itemize}
    \item Venue ID
    \item Venue category name
    \item Coordinates
    \item Country code
\end{itemize}


Foursquare check-ins were merged with Foursquare \acrshort{poi} by the venue ID to get the coordinates of venues. The merged Foursquare data should be further cleaned and differentiated as either locals or tourists, and the preprocessing steps were as follows:
\begin{enumerate}
    \item Merged check-ins with \acrshort{poi} to get the coordinates for each check-in.
    \item Removed duplicated check-ins.
    \item Updated venue categories of check-ins based on the Foursquare category \footnote{\url{http://foursquare-categories.herokuapp.com/}}. The Foursquare category system uses a three-level hierarchy. For example, the Arts \& Entertainment first-level category includes a second-level category called Movie Theater, which in turn contains three third-level categories: Drive-in Theater, Indie Movie Theater, and Multiplex. This study kept only the first-level categories to represent 
    check-ins.
    \item Removed check-ins that were labeled as Residence to protect the privacy of Foursquare users.
    \item Removed users whose total travel distances are less than one kilometer. This study aims to investigate city perception through the analysis of trajectories. As such, trajectories with short travel distances are considered unsuitable for discovering meaningful patterns of city perception. Therefore, users with short travel distances were removed.
    \item Identified locals and tourists based on the number of days spent in Greater London, total travel distances, and user profiles. Users who spent more than 30 days in Greater London and had a total travel distance greater than 100 km were considered locals. For the remaining users, those who listed London as their hometown in their profiles were also categorized as locals. The user profiles were retrieved using Get User Details \footnote{\url{https://location.foursquare.com/developer/reference/v2-users-user_id}} provided by Foursquare API.
\end{enumerate}

After all these steps, a total of 177,207 check-ins by 7,183 users in Greater London were left. Out of these users, 1,087 were identified as locals and they posted 109,558 check-ins, which was far more than the number of check-ins posted by tourists, who amounted to 6,096, but posted only 67,649 check-ins (Table~\ref{tab:foursquare_summary}). Figure~\ref{fig:foursquare_distribution} shows the distribution of Foursquare check-ins, and most check-ins were concentrated in the central area of Greater London. Moreover, compared with tourists, locals tended to share more check-ins in peripheral areas.

\begin{table}[h!]
\centering
\caption{\label{tab:foursquare_summary}Summary of Foursquare data.}
\begin{tabular}{llll} \hline
Time & Population group & No. of users & No. of check-ins \\
\hline
\multirow{3}{*}{Overall} 
& All Users & 7,183 & 177,207 \\
& Locals & 1,087 & 109,558 \\
& Tourists & 6,096 & 67,649 \\
\hline
\multirow{3}{*}{Daytime} 
& All Users & 7,055 & 141,353 \\
& Locals & 1,082 & 87,803 \\
& Tourists & 5,973 & 53,550 \\
\hline
\multirow{3}{*}{Nighttime} 
& All Users & 4,932 & 33,256 \\
& Locals & 1,018 & 20,469 \\
& Tourists & 3,914 & 12,787 \\
\hline
\multirow{3}{*}{Weekday} 
& All Users & 6,692 & 128,797 \\
& Locals & 1,072 & 81,677 \\
& Tourists & 5,620 & 47,120 \\
\hline
\multirow{3}{*}{Weekend} 
& All Users & 5,174 & 45,812 \\
& Locals & 1,022 & 26,595 \\
& Tourists & 4,152 & 19,217 \\
\hline
\end{tabular}
\end{table}


\begin{figure}[h!]
\centering
\includegraphics[width=1\textwidth]{figures/foursquare_distribution.png}
\caption{\label{fig:foursquare_distribution}Distribution of Foursquare check-ins of locals and tourists.}
\end{figure}

For venue categories, Foursquare categorizes the place types into ten categories, namely (1) Travel \& Transport, (2) Food, (3) Professional \& Other Places, (4) Outdoors \& Recreation, (5) Shop \& Service, (6) Nightlife Spot, (7) Arts \& Entertainment, (8) College \& University, (9) Event, (10) Residence. Category 9 Event was not contained in the check-ins used in this study, and Category 10 Residence was removed to protect the privacy of users. The number of check-ins for each Foursquare venue category is shown in Table~\ref{tab:foursquare_category}. There are several combined categories in the Foursquare category system, such as Professional \& Other Places and Outdoors \& Recreation, which makes it ambiguous to determine the visiting purposes of individual check-ins. To overcome this limitation and investigate place properties more precisely, this study divided venues into 11 categories by separating the combined categories and defining new ones based on the third-level categories in the Foursquare category system. The number of check-ins for modified venue categories is presented in Table~\ref{tab:modified_category}. In terms of the category distribution among locals and tourists, the categories Transport and Restaurant were the most prevalent among both locals and tourists. However, certain categories showed a higher prevalence among a specific group, for instance, Shopping Place, Art Place, and Accommodation were more welcomed by tourists, whereas Sports Place received a greater number of check-ins from locals (Figure~\ref{fig:foursquare_category}).

\begin{table}[h!]
\centering
\caption{\label{tab:foursquare_category}Foursquare venue category.}
\begin{tabular}{lll} \hline
No. & Foursquare venue category & No. of check-ins \\ \hline
1 & Travel \& Transport & 43,051 \\
2 & Food & 34,376 \\
3 & Professional \& Other Places & 22,184 \\
4 & Outdoors \& Recreation & 20,429 \\
5 & Shop \& Service & 19,772 \\
6 & Nightlife Spot \& Service & 18,418 \\
7 & Arts \& Entertainment & 14,285 \\
8 & College \& University & 4,692 \\ \hline
\end{tabular}
\end{table}


\begin{table}[h!]
\centering
\caption{\label{tab:modified_category}Modified Foursquare venue category.}
\begin{adjustbox}{max width=\textwidth, margin=0cm}
\begin{tabular}{lllp{6cm}p{5cm}} \hline
No. & Modified venue category & No. of check-ins & Foursquare venue category 
& Example \\ \hline
1 & Transport & 36,994 & Travel \& Transport & Airport, Train Station, Bus Station, etc. \\
2 & Restaurant & 34,376 & Food & Sandwich Place, Asian Restaurant, Italian Restaurant, etc. \\
3 & Professional Place & 29,118 & Professional \& Other Places, College \& University & Office, Government Building, Police Station, etc. \\
4 & Entertainment Place & 22,434 & Arts \& Entertainment, Shop \& Service, Nightlife Spot & Casino, Bowling Alley, Zoo, Bar, Nightclub, etc. \\
5 & Shopping Place & 18,127 & Shop \& Service & Outlet Mall, Clothing Store, Market, etc. \\
6 & Art Place & 11,132 & Arts \& Entertainment, Outdoors \& Recreation & Museum, Movie Theater, Art Gallery, Music Venue, etc. \\
7 & Green \& Blue Space & 8,220 & Outdoors \& Recreation & Park, Plaza, Garden, Lake, etc. \\
8 & Accommodation Place & 5,902 & Travel \& Transport & Hotel \\
9 & Sports Place & 5,852 & Outdoors \& Recreation & Athletics \& Sports, Pool, Playground, etc. \\
10 & Others & 3,346 & Outdoors \& Recreation, Professional \& Other Places, Travel \& Transport & Building, Bridge, Well, etc. \\
11 & Service Place & 1,706 & Travel \& Transport, Shop \& Service & Bank, Drugstore, Car Wash, Laundry Service, etc. \\ \hline
\end{tabular}
\end{adjustbox}
\end{table}


\begin{figure}[h!]
\centering
\includegraphics[width=1\textwidth]{figures/foursquare_category.png}
\caption{\label{fig:foursquare_category}Number of venue categories visited by locals and tourists.}
\end{figure}



\subsubsection{Flickr Data}
Flickr is an online photo management and sharing application where users can upload photos. As of 2023, the Flickr community has shared tens of billions of photos. Flickr data was collected via Flickr API with the Python library flickrapi \footnote{\url{https://stuvel.eu/software/flickrapi/}}. To collect Flickr photos, the method flickr.photos.search \footnote{\url{https://www.flickr.com/services/api/flickr.photos.search.html}} was applied to get photos. A total of 954,260 photos with unique 250,254 tags uploaded by 28,633 users in Greater London from April 3, 2012 to September 16, 2013 were collected. The main fields of Flickr data include:
\begin{itemize}
    \item Photo ID
    \item User ID
    \item Date taken
    \item Accuracy
    \item Coordinates
    \item Tags
    \item Title
    \item Photo URL
    \item Number of views
\end{itemize}

The Flickr data required additional cleaning and identification of locals and tourists. The preprocessing steps were as follows:
\begin{enumerate}
    \item Removed photos without tags because Flickr tags were one of the main semantic information in this study.
    \item Removed photos with an accuracy level below 14. Flickr divides the accuracy level on a scale of 1 to 16, with 1 representing world-level accuracy, 2-3 representing country-level accuracy, 4-6 representing region-level accuracy, 7-11 representing city-level accuracy, and 12-16 representing street-level accuracy. Only photos with an accuracy greater than 14 were kept in this study.
    \item Removed tags that were non-Ascii characters, special characters, numbers, stop-word (e.g., a, an, the), prepositions (e.g., from, to), general place names (e.g., britain, uk, england, london), and other irrelevant tags (e.g., nikon, samsung, instagram, flickr).
    \item Removed duplicates. Duplicated tags in the same list were removed. Moreover, duplicated photos of the same user at the same location with the same tags were removed.
    \item Removed prolific and unprolific users. Prolific users were defined as those who contributed more than 5\% of all photos in the dataset, while unprolific users were defined as those who uploaded less than five photos in total per day.
    \item Removed tags with a coefficient of variation greater than 300 to reduce the contribution bias. Some tags were dominantly used by prolific users, which might lead to distortions of tags. The coefficient of variation measures whether a tag is evenly used among prolific and unprolific users \cite{hollenstein_exploring_2010}. To determine the coefficient of variation, the Flickr photos were sorted based on user contribution, with the most prolific users' photos at the top. Subsequently, all photos were equally distributed into 100 bins. For each tag, a tag profile was constructed, which stored its frequency in each of the 100 bins. The coefficient of variation was calculated as the ratio of the standard deviation to the mean of the tag frequencies in the 100 bins. The lower the coefficient of variation, the more evenly the tag is distributed. For instance, based on the histogram that displays the number of photos with a certain tag in a user distribution order, the tag \textit{square} with a coefficient of variation less than 300 is more evenly distributed than the tag \textit{forms} with a coefficient of variation greater than 300.
    \item Removed tags that were place names in Greater London. GeoNames offers Free Gazetteer Data \footnote{\url{http://download.geonames.org/export/dump/}} that includes place names in Great Britain. Place names in Greater London were filtered out from the downloaded data, and any tag that matched with these place names was removed.
    \item Identified locals and tourists based on the number of days spent in Greater London and user profiles. This study used flickr.profile.getProfile \footnote{\url{https://www.flickr.com/services/api/flickr.profile.getProfile.html}} provided by Flickr API to collect user profiles, which contained information about the users' hometowns and cities. Users who had stayed in Greater London for more than 30 days were considered locals. For the remaining users, this study would also flag them as locals if their hometown or city in their profiles was listed as London.
\end{enumerate}

\begin{figure}[h!]
\centering
\includegraphics[width=1\textwidth]{figures/flickr_tags_cv.png}
\caption{\label{fig:flickr_tags_cv}Tag profiles.}
\end{figure}

After all these steps, 177,568 photos were left with a total of 717,747 tags (3,365 unique tags). These tags were used by 4,687 users, with 851 locals and 3,836 tourists uploading 259,967 and 457,780 tags respectively (Table~\ref{tab:flickr_summary}). The distribution of Flickr photos indicates that the majority of photos were taken in the center of Greater London, while tourists tended to share more photos around Heathrow Airport (Figure~\ref{fig:flickr_distribution}). The word cloud in Figure~\ref{fig:flickr_tag_cloud} illustrates the most frequently used tags, including city, architecture, street, park, and art.


\begin{table}[h!]
\centering
\caption{\label{tab:flickr_summary}Summary of Flickr data.}
\begin{tabular}{lllll} \hline
Time & Population group & No. of users & No. of photos & No. of tags \\
\hline
\multirow{3}{*}{Overall} 
& All Users & 4,687 & 177,568 & 717,747 \\
& Locals & 851 & 65,403 & 259,967 \\
& Tourists & 3,836 & 112,165 & 457,780 \\
\hline
\multirow{3}{*}{Daytime} 
& All Users & 4,254 & 144,748 & 580,753 \\
& Locals & 768 & 52,908 & 206,737 \\
& Tourists & 3,486 & 91,840 & 374,016 \\
\hline
\multirow{3}{*}{Nighttime} 
& All Users & 2,372 & 32,820 & 136,994 \\
& Locals & 479 & 12,495 & 53,230 \\
& Tourists & 1,893 & 20,325 & 83,764 \\
\hline
\multirow{3}{*}{Weekday} 
& All Users & 3,307 & 94,167 & 378,574 \\
& Locals & 618 & 34,243 & 132,779 \\
& Tourists & 2,689 & 59,924 & 245,795 \\
\hline
\multirow{3}{*}{Weekend} 
& All Users & 2,755 & 83,401 & 339,173 \\
& Locals & 592 & 31,160 & 127,188 \\
& Tourists & 2,163 & 52,241 & 211,985 \\
\hline
\end{tabular}
\end{table}

\begin{figure}[h!]
\centering
\includegraphics[width=1\textwidth]{figures/flickr_distribution.png}
\caption{\label{fig:flickr_distribution}Distribution of Flickr photos of locals and tourists.}
\end{figure}

\begin{figure}[h!]
\centering
\includegraphics[width=0.7\textwidth]{figures/flickr_tag_cloud.png}
\caption{\label{fig:flickr_tag_cloud}Flickr tag cloud.}
\end{figure}


\clearpage


% ============================================ ||| Methodology ||| ============================================
\section{Methodology}

% ============================= || Overview || =============================
\subsection{Overview}
The methodology employed in this study is illustrated in Figure~\ref{fig:workflow}. To explore individuals' perception of the city, Foursquare check-ins and Flickr tags were utilized, and the process of data preprocessing is elaborated in Section \ref{data_preprocessing}. This workflow is designed to address two research questions. The first research question was tackled by employing the methods discussed in Section \ref{hotspots_detection}, which involved identifying hotspots of locals and tourists using \acrfull{kde} based on their Foursquare check-ins. To investigate the popularity of specific areas among locals or tourists, the difference ratio between these two groups of people was calculated and visualized in rasters. For the second research question, the construction of semantic trajectories was necessary. Prior to the trajectory construction, the Place was modeled and enriched with semantic information based on its dimensions (Section \ref{place_modeling}). Subsequently, semantic trajectories were clustered, and typical semantic trajectories from each cluster were extracted through sequential pattern mining (Section \ref{typical_semantic_trajectory_mining}). This study was conducted based on Python 3.10.12.


\begin{figure}[h!]
\centering
\includegraphics[width=1\textwidth]{figures/workflow.png}
\caption{\label{fig:workflow}Workflow.}
\end{figure}


% ============================= || Hotspots Detection || =============================
\subsection{Hotspots Detection}\label{hotspots_detection}

% ====================== | Kernel Density Estimation | ======================
\subsubsection{Kernel Density Estimation}
Differences in visiting preferences between locals and tourists show the importance of investigating the distribution patterns within these distinct population groups. To explore these patterns, the use of a nonparametric model called \acrfull{kde} proves valuable. \acrshort{kde} offers a means of estimating probability distributions without relying on specific assumptions about their form. Unlike parametric approaches, nonparametric density estimation requires a minimum number of parameters to construct a model and estimates these parameters through the likelihood principle. This approach provides flexibility in modeling the distribution of data. To formally define \acrshort{kde}, consider a set of independent and identically distributed random samples, denoted as $X_{1},X_{2},...,X_{n} \in \mathbb{R}$, drawn from an unknown distribution $P$ with density function $p$. In this context, \acrshort{kde} can be expressed as:

\begin{equation} \label{eq:kde}
\hat{p}_{n}(x) = \frac{1}{nh^{d}} \sum_{i=1}^n k\bigg(\frac{x-X_i}{h}\bigg)
\end{equation}

where $h$ is a bandwidth parameter that controls the degree of smoothing in \acrshort{kde}. A large bandwidth might result in an over-smoothed estimate, obscuring meaningful features within the data. Conversely, a small bandwidth might lead to an under-smoothed estimate, failing to reveal the true underlying shape with random noise. The kernel function $K: \mathbb{R^{d}} \rightarrow \mathbb{R}$ is a smooth function employed by \acrshort{kde} to calculate the weighted average of neighboring observed data. This weighting process effectively smooths out the data, with closer points receiving higher weights. The Gaussian kernel (Equation \ref{eq:gaussian_kernel}) and the Spherical kernel (Equation \ref{eq:spherical_kernel}) are the two widely used kernels.

\begin{equation} \label{eq:gaussian_kernel}
K_{\text{Gaussian}}(x) = \frac{exp\left(-\frac{\|x\|^{2}}{2}\right)}{v_{1,d}}, \quad v_{1,d} = \int exp\left(-\frac{\|x\|^{2}}{2}\right) \, dx 
\end{equation}

\begin{equation} \label{eq:spherical_kernel}
K_{\text{Spherical}}(x) = \frac{I\left(\|x\| \leq 1\right)}{v_{2,d}}, \quad v_{2,d} = \int I\left(\|x\| \leq 1\right) \, dx 
\end{equation}

In this study, \acrshort{kde} was employed to analyze the distribution of Foursquare check-ins of locals and tourists. To visualize the distribution effectively, the seaborn.kdeplot \footnote{\url{https://seaborn.pydata.org/generated/seaborn.kdeplot.html}} function from the Python library seaborn was utilized. The default bandwidth method \textit{scoott}, with a default bandwidth value of 1 was applied. This choice is appropriate given the smooth distribution of Foursquare check-ins, making the default bandwidth value well-suited for the analysis. Furthermore, the Gaussian kernel was provided by seaborn.kdeplot was used as the kernel smoother.

% ====================== | Calculation of Difference Ratio | ======================
\subsubsection{Calculation of Difference Ratio}
The extent to which locals and tourists mix within a given area provides insights into the popularity of that area among these two distinct groups. Representing the mixture of these two groups based solely on the count difference between locals and tourists may introduce bias, as the significant disparity in their numbers can skew the results. To assess the level of integration between locals and tourists, the dissimilarity index $D$ was employed, as previously utilized in a study by \cite{li_analyzing_2018}. The dissimilarity index quantifies the degree of dissimilarity between the two groups and is defined as:

\begin{equation} \label{eq:dissimilarity_index}
    D = \frac{1}{2}\sum_{i=1}^{N}\bigg|\frac{t_{i}}{T}-\frac{l_{i}}{L}\bigg|
\end{equation}

where $t_{i}$ represents the number of tourists in the $i_{th}$ given area, and $T$ represents the total number of tourists in the city. Similarly, $l_{i}$ represents the number of locals in the $i_{th}$ given area, and $L$ represents the total number of locals in the city. A higher value of $D$ indicates a greater degree of mixing between locals and tourists across the entire area, and conversely, a lower value suggests a lower level of mixing. However, it's important to note that the $D$ provides only a single value representing the overall mixture degree within the city, lacking information about the distribution of the mixture degree across different areas in the city.

To address this limitation, this study employed a rasterization approach, dividing Greater London into equal-sized cells with a cell size of 1km by 1 km. This allowed for the visualization of the mixture degree within each raster. While both the number of visitors and the number of check-ins in each raster could be used to calculate the mixture degree, the latter is more suitable for reflecting activity density. Consequently, the number of check-ins made by locals and tourists in each raster was counted respectively to calculate the mixture degree. To capture the mixture degree in each raster, the difference ratio $R$ was computed based on the equation of dissimilarity index $D$. This ratio serves as a representative measure of the mixture degree within each raster, and it is defined as:

\begin{equation} \label{eq:diff_ratio}
    R = \frac{t_{i}}{T}-\frac{l_{i}}{L}
\end{equation}

where $t_{i}$ represents the number of tourists' check-ins in the raster $i$, and $T$ represents the total number of tourists' check-ins shared citywide. Similarly, $l_{i}$ represents the number of locals' check-ins in the raster $i$, and $L$ represents the total number of locals' check-ins shared citywide. This ratio provides insights into the relative activity density between locals and tourists within each raster. A larger positive value of $R$ indicates a higher density of check-ins shared by tourists in that particular raster. Conversely, a larger negative value of \(R\) indicates a higher density of check-ins shared by locals in the same raster.

% ============================= || Place Modeling || =============================
\subsection{Place Modeling} \label{place_modeling}

% ====================== | Place Construction | ======================
\subsubsection{Place Construction} \label{place_construction}
A place is a segment of geographical space that holds meaningful attributes, while a check-in is a specific instance or event that represents an individual's interaction with a place. In this study, places were constructed by clustering check-ins using the \acrfull{hdbscan} \citep{campello_density-based_2013}. \acrshort{hdbscan} is a hierarchical extension of \acrshort{dbscan} that automatically determines the optimal number of clusters and can handle clusters of varying densities, reducing user input bias. Prior to \acrshort{hdbscan}, the Gower distance \citep{gower_general_1971} was computed as a distance matrix. The Gower distance is suitable for measuring the dissimilarity between items with mixed numeric and non-numeric data. The Gower similarity matrix is defined as:

\begin{equation} \label{gower_similarity}
S_{Gower}(x_{i},x_{j}) = \frac{\sum_{k=1}^{p}s_{ijk}\delta_{ijk}}{\sum_{k=1}^{p}\delta_{ijk}}
\end{equation}

where $s_{ijk} \in [0,1]$ is the score for each feature $k = 1,2, ...,p$. If $x_{i}$ and $x_{j}$ are close to each other along feature $k$, the score $s_{ijk}$ is close to 1. Conversely, if they are far apart along feature $k$, the score $s_{ijk}$ is close to 0. The Gower similarity matrix was then converted to a distance matrix:

\begin{equation} \label{gower_distance}
D_{Gower}(x_{i},x_{j}) = \sqrt{1-S_{Gower}(x_{i},x_{j})}
\end{equation}

The features used for clustering included the coordinates and categories of check-ins. The coordinates of check-ins were converted to EPSG:27700 projected coordinate system for the United Kingdom, and then standardized by the StandardScaler to unit variance. The categories of check-ins were converted into numerical values using label encoding. The feature importance was considered in the distance matrix, with coordinates accounting for 80\% of the weighting and categories accounting for 20\%. 

In the application of \acrshort{hdbscan}, a minimum number of three check-ins within a cluster was set to define a place, ensuring that each place encompassed a certain number of check-ins. Check-ins labeled as -1 were considered outliers and removed from the study. To represent the spatial extent of a place, convex hulls were built around the clusters of check-ins, creating polygons that approximate the area covered by each place.

To enrich the semantic information of places, three dimensions conceptualized by \citep{agnew_space_2011}, namely location, locale, and sense of place, were employed. The location dimension was represented by the borough name, specifically the borough where the majority of check-ins within a place cluster were located. The locale dimension was represented by the most frequently occurring check-in category within the place cluster. The sense of place dimension utilized Flickr tags associated with the place cluster to capture people's perceptions. Topic modeling was applied to generate a set of topics based on these tags, providing a representation of the sense of place associated with each place.

% ====================== | Topic Modeling | ======================
\subsubsection{Topic Modeling}
\acrfull{lda} was employed for topic modeling in this study. \acrshort{lda} is a three-level hierarchical Bayesian model in which words (e.g.,, tags) are collected into documents, and each word's presence is attributable to one of the document's topics. The generative process of \acrshort{lda} is illustrated in Figure~\ref{fig:lda}. To generate a document, a K-vector $\theta$ representing the mixture proportion of $k$ topics is sampled from a Dirichlet prior distribution $p(\theta|\alpha)$. The variable $k$ specifies the dimension of the distribution and also the dimension of the topic variable $z$, which defines the total number of topics generated by the model. The Dirichlet variable $\theta$ takes values within the $(k-1)$-simplex and has a probability density on this simplex, defined by the following equation:

\begin{equation} \label{eq:lda}
    p(\theta|\alpha)\ = \frac{\Gamma(\sum_{i=1}^{k} \alpha_{i})}{\prod_{i=1}^{k}\Gamma(\alpha_{i})}\theta_{1}^{z_{1}-1}...\theta_{k}^{z_{k}-1}
\end{equation}

\begin{figure}[!h]
\centering
\includegraphics[width=0.6\textwidth]{figures/lda.png}
\caption{\label{fig:lda}Illustration of the LDA generative process. The outer box M denotes the repeated sampling for each document and the inner box N represents sampling within each document. The box K represents sampling for each topic \citep{vayansky_review_2020}.}
\end{figure}

where $\alpha$ is the hyperparameter of the Dirichlet distribution, which represents a prior count of the frequency of an individual topic appearing in a document. The parameter $\beta$ is also a Dirichlet prior hyperparameter that determines the smoothing level over word distributions in all topics. The values of $\alpha$ and $\beta$ are selected based on the size of the vocabulary and the number of topics chosen \citep{blei_latent_2003}. To determine the optimal number of topics, the \textit{coherence value} can be used to measure the semantic quality of the generated topics. The \textit{coherence value} is calculated based on the probability of words in a topic co-occurring in the place clusters assigned that topic, and it can be defined as:

\begin{equation} \label{eq:coherence_value}
    coherence = \sum_{i}\sum_{j<1}\log\frac{D(w_{j},w_{i})+\beta}{D(w_{i})}
\end{equation}

where $\beta$ is to prevent log zero errors, $D(w_{j},w_{i})$ is the number of co-occurrence of two terms in a document, and $D(w_{i}$ is the number of occurrences of the more probable terms. A higher coherence value indicates that the words within a topic are more closely related to each other, making topics more interpretable to humans.

In practice, the \acrshort{lda} model was implemented using \acrfull{mallet}. \acrshort{mallet} is a Java-based console application, but the Python library gensim, developed by \cite{rehurek_software_2010}, provides a wrapper model called LdaMallet for \acrshort{mallet}, which was utilized in this study. The input to the \acrshort{lda} model was a corpus consisting of documents, where each document stored the Flickr tags within each place cluster represented by the place convex hull. The corpus was structured as a $word \times document$ matrix, where each row represented a vector $w = {w_{1},w_{2},...,w_{v}}$ of length $V$ corresponding to the number of words in the vocabulary, and $w_{v}$ denoted the frequency of word $v$ in the document. The \acrshort{lda} model generated a set of topics as its main output, along with topic distribution and topic terms, which facilitated the determination and interpretation of topics for each place. The topic distribution for each place consisted of a list of topics and the probabilities of the place belonging to each corresponding topic. The topic terms for each topic comprised a list of tags, accompanied by their probabilities of being associated with that topic. The most probable topic was assigned to each place, and the tags for each topic were visualized using LDAvis to explore the distribution of tags within topics.

In instances where certain places had no tags within their convex hulls, a topic imputation was employed to assign topics to those empty places. This imputation process relied on the topic distribution of nearby places that did contain Flickr tags in a weighted average approach. The imputed topic distribution represented the probability distribution across a predefined set of topics. Based on this imputation topic distribution, the places were assigned the most probable topics. The calculation for imputed topic distribution of the target empty place can be expressed as:

\begin{equation} \label{eq:topic_imputation}
    \theta_{imputed} = \frac{1}{W}\sum_{i}{n}w_{i}\theta_{i}
\end{equation}

where $\theta_{i}$ represents the topic distribution for place $i$, and $n$ is the number of places in the neighborhood of the target empty place. The weight assigned to each place, $w_{i}$, is determined by the distance between place $i$ and the target empty place, which is defined as:

\begin{equation} \label{eq:topic_imputation_w_i}
    w_{i} = \exp(\frac{-d(i,j)^2}{2\sigma^2})
\end{equation}

where $\sigma$ controls the smoothness of the weighting function. $d(i,t)$ is the distance between place $i$ and place $j$. A larger value of $\sigma$ gives more weight to places that are closer to the target empty place, while a smaller value of $\sigma$ gives more weight to places that are farther away. $W$ ensures that the sum of the weights to 1 so that the imputed topic distribution is a valid probability distribution, which is defined as:

\begin{equation} \label{eq:topic_imputation_W}
    W = \sum_{i}{n}w_{i}
\end{equation}




% ============================= || Typical Semantic Trajectory Mining || =============================
\subsection{Typical Semantic Trajectory Mining} \label{typical_semantic_trajectory_mining}

% ====================== | Semantic Trajectory Construction | ======================
\subsubsection{Semantic Trajectory Construction}
This study investigated city perception by analyzing semantic trajectories. These trajectories were constructed based on the places defined in Section \ref{place_construction}. The concept of semantic trajectories is defined in Definition \ref{def:semantic_trajectory}.

\begin{definition} \label{def:semantic_trajectory}
A semantic trajectory is a sequence of places $T=\langle p_{1},p_{2},...,p_{n} \rangle$, with $p=(x,y,D_{p})$ being the place $i$ at the position $(x,y)$ annotated by the three dimensions of the place.
\end{definition}

To determine the beginning and end of each trajectory, it was necessary to consider the movement of individuals, as shown in Figure~\ref{fig:trajectory_begin_end}. A time gap of five hours was used as a criterion to determine whether two consecutive places belonged to the same trajectory. Specifically, if the time gap between two consecutive places exceeded five hours, the former place was considered the last position of the current trajectory, while the latter place marked the beginning of the new trajectory. Furthermore, the analysis of city perception focused on long trajectories as they conveyed more information. Therefore, only trajectories consisting of more than five places were considered in this study.

\begin{figure}
\centering
\includegraphics[width=0.75\textwidth]{figures/begin_end.png}
\caption{\label{fig:trajectory_begin_end}Trajectories extracted from a movement track \citep{parent_semantic_2013}.}
\end{figure}


% ====================== | Semantic Trajectory Similarity Measure | ======================
\subsubsection{Semantic Trajectory Similarity Measure} \label{traj_similarity}
Prior to clustering semantic trajectories, it is crucial to measure their similarity. Many existing methods focus on similarity measuring of raw trajectories, which only consider spatial or spatiotemporal aspects. This study investigated semantic trajectories enriched with three dimensions, therefore, it is necessary to consider multiple dimensions in the similarity measurement. \acrfull{msm} proposed by \cite{furtado_multidimensional_2016} was employed to measure trajectory similarity in this study. 

\acrshort{msm} is a similarity measure for multidimensional sequences, considering and weighing the similarity across all dimensions. This study considered the spatial position dimension and the three semantic dimensions of places. These dimensions are defined in Definition \ref{def:semantic_dimension}.

\begin{definition} \label{def:semantic_dimension}
In two semantic trajectories, denoted as $T_{1}$ and $T_{2}$, the general term for each dimension $D_{k}$ (where $k=1,2,3,4$) represents the following: $D_{1}$ refers to the spatial position of the semantic trajectory, $D_{2}$, $D_{3}$, and $D_{4}$ represent the three semantic dimensions, namely location, locale, and sense of place. The distance between any two places, $p_{1} \in T_{1}$ and $p_{2} \in T_{2}$, is computed using the distance function $dist_{k}(p_{1},p_{2})$. Additionally, a maximum distance threshold $maxDist_{k}$ is set for each dimension to determine whether a pair of places $(p_{1},p_{2})$ can be considered a match in $D_{k}$.
\end{definition}

In the computation of the similarity score, it is possible for $p_{1}$ and $p_{2}$ to be matched in one dimension but not in another. Therefore, the dimensions are considered separately. Furthermore, to account for the relative importance of each dimension, a weight $w_{k}$ is assigned to represent its significance. It should be noted that the sum of the weights of all dimensions should be equal to 1. In this study, the weights were set as follows: $w_{1}=0.4$, $w_{2}=0.1$, $w_{3}=0.3$, $w_{4}=0.2$. The matching score between the two places $p_{1}$ and $p_{2}$ can be defined as the sum of the weighted match scores across all dimensions, as represented in the following equation:

\begin{equation} \label{eq:match_score}
    score(p_{1},p_{2}) = \sum_{k=1}^{4}(match_{k}(p_{1},p_{2})w_{k})
\end{equation}

where $match_{k}(p_{1},p_{2})$ determines whether the places are matched or not based on the max distance threshold $maxDist_{k}$, and it is defined as:

\begin{equation} \label{eq:match}
    match_{k}(p_{1},p_{2}) = \begin{cases}
    1 & \text{if } dist_{k}(p_{1},p_{2}) \leq maxDist_{k}  \\
    0 & \text{otherwise}
    \end{cases}
\end{equation}

Since a place $p_{1} \in T_{1}$ can be matched with multiple places in $T_{2}$, the aim of \acrshort{msm} is to find the best matching score for each place $p_{1}$ with $T_{2}$. The parity of $T_{1}$ with $T_{2}$, denoted as $parity(T_{1},T_{2})$, is defined as the sum of the highest score of all places $p_{1} \in T_{1}$ with $T_{2}$, and the equation is as follows:

\begin{equation} \label{eq:parity}
    parity(T_{1},T_{2}) = \sum_{p_{T{1}}\in T_{1}}\max\{score(p_{1},p_{2}) : p_{T{2}} \in T_{2}\}
\end{equation}

Finally, the multidimensional similarity measure $MSM(T_{1},T_{2})$ is calculated by averaging the parity of $T_{1}$ with $T_{2}$ and and the parity of $T_{2}$ with $T_{1}$, which is defined as:

\begin{equation} \label{eq:msm}
    MSM(T_{1},T_{2}) = \begin{cases}
    0 & \text{if } |T_{1}| = 0 \text{or} |T_{2}| = 0 \\
    \frac{parity(T_{1},T_{2}) + parity(T_{2},T_{1})}{|T_{1}|+|T_{2}|} & \text{otherwise}
    \end{cases}
\end{equation}

To handle the different data types of dimensions, each dimension was assigned a corresponding distance function and a threshold to determine whether $p_{1}$ and $p_{2}$ were matched. For the spatial position dimension $D_{1}$, the Euclidean distance was applied (Equation~\ref{eq:dist_euclidean}), where $x$ and $y$ represented the coordinates of places. The maximum distance threshold $maxDist_{1}$ was set to 1000 meters, meaning that if the distance between $p_{1}$ and $p_{2}$ exceeded 1000 meters, the two points were not considered a match. For the semantic dimensions $D_{2}$, $D_{3}$, and $D_{4}$, which were categorical data, the discrete distance was applied (Equation~\ref{eq:dist_discrete}). In this case, the distance could only be either 0 or 1, and the maximum distance thresholds $maxDist_{2}$, $maxDist_{3}$, and $maxDist_{4}$ for these dimensions were set to 0.5.

\begin{equation} \label{eq:dist_euclidean}
    dist_{euclidean}(p_{1},p_{2}) = \sqrt{(p_{1}.x-p_{2}.x)^2+(p_{1}.y-p_{2}.y)^2}
\end{equation}

\begin{equation} \label{eq:dist_discrete}
    dist_{discrete}(p_{1},p_{2}) = \begin{cases}
    0 & \text{if } p_{1}.type = p_{2}.type  \\
    1 & \text{otherwise}
    \end{cases}
\end{equation}

In practice, the Python library trajminer developed by \cite{petry_trajminer_2019} was applied to construct the similarity matrix for semantic trajectories. Specifically, the function trajminer.similarity.MSM for \acrfull{msm} was used, with the distance functions, thresholds, and weights specified. Overall, by employing the \acrshort{msm} approach, this study took into account the multiple dimensions of semantic trajectories and computed a similarity matrix that captured the matching scores and weights for each dimension, providing a comprehensive assessment of trajectory similarity.



% ====================== | Semantic Trajectory Clustering | ======================
\subsubsection{Semantic Trajectory Clustering}
The K-medoids algorithm was employed to cluster semantic trajectories. The term K-medoids was coined by \cite{kaufman_partitioning_1990} with their Partitioning Around Medoids (PAM) algorithm. K-medoids is a classical clustering technique that divides the dataset of $n$ objects into $k (k<n)$ clusters, assuming a predetermined number of clusters $k$. Similar to K-means, K-medoids aims to minimize the distance between data points assigned to a cluster and a designated point as the cluster's center. However, unlike K-means, K-medoids minimizes a sum of pairwise dissimilarities rather than a sum of squared Euclidean distances, and it selects the data point as the center of the cluster.

In the K-medoids algorithm, the dissimilarity between trajectories is measured, which is computed based on the trajectory similarity as described in Section \ref{traj_similarity}. The trajectory dissimilarity between each pair of trajectories is defined as:

\begin{equation} \label{eq:dissimilarity}
    dissimilarity(T_{1},T_{2}) = 1-MSM(T_{1},T_{2})
\end{equation}

To perform the clustering for semantic trajectories, the Python library trajminer's function trajminer.clustering.KMedoids was employed. With the precomputed dissimilarity matrix, the initial cluster medoids were determined using the approach introduced by \cite{park_simple_2009}. Simplifying $dissimilarity(T_{i},T_{j})$ as $d_{ij}$ to represent the dissimilarity between trajectory $i$ and trajectory $j$, the value $v_{j}$ for trajectory $j$ is calculated as:

\begin{equation} \label{eq:vj}
    v_{j} = \sum_{i=1}^{n}\frac{d_{ij}}{\sum_{l=1}^{n}d_{il}}, j=1,2,...,n
\end{equation}

Given the number of clusters $k$, the initial medoids are determined by selecting the $k$ trajectories with the first $k$ smallest $v_{j}$ values. The remaining trajectories are then assigned to their nearest medoids to construct initial clusters. Next, the sum of distances from all trajectories to their medoids is calculated to update the medoids. This process iteratively finds a new medoid for each cluster with a minimum total distance to other trajectories within the cluster, replacing the current medoid with the new one. Subsequently, the assignment of trajectories to the nearest medoids is updated, and this iterative process continues until the sum of distances from all trajectories to their medoids remains unchanged.

The clustering results can be evaluated using the silhouette score, which measures the similarity of trajectory to its own cluster compared to other clusters. The score ranges from -1 to 1, with a score closer to 1 indicating a well-matched trajectory within its own cluster and a poor match to neighboring clusters. In this study, the silhouette score was employed to determine the optimal number of clusters.

% ====================== | Sequential Pattern Mining | ======================
\subsubsection{Sequential Pattern Mining}
Finding typical semantic trajectories in each cluster helps to investigate city perception, as these typical trajectories serve as representatives of their respective clusters. This study applied sequential pattern mining using the \acrfull{prefixspan} algorithm to detect frequently visited sequences. The \acrshort{prefixspan}, proposed by \cite{pei_mining_2004}, is an efficient method for sequential pattern mining. It follows a projection-based and sequential pattern-growth approach, offering ordered growth and reduced projected databases. The fundamental concept of the \acrshort{prefixspan} is to recursively project the sequence databases into smaller projected databases based on the current sequential patterns. Let $I=\{i_{1},i_{2},...,i_{n}\}$ be a set of all items. An itemset is a subset of items. A sequence $s$ is an ordered list of itemsets, which is denoted by $\langle s_{1}s_{2}...s_{l} \rangle$, where $s_{j}$ represents an itemset. $s_{j}$ can also be seen as an element of the sequence, denoted as $(x_{1}x_{2}...x_{m})$, where $x_{k}$ is an item. An $l$-sequence refers to a sequence with a length of $l$. A subsequence $\alpha=\langle a_{1}a_{2}...a_{n} \rangle$ is defined as a sequence that appears inits supersequence $\beta=\langle b_{1}b_{2}...b_{n} \rangle$ $(\alpha \subseteq \beta)$. A sequence database $S$ is a collection of tuples $\langle sid,s \rangle$, where $sid$ represents the sequence id of sequence $s$. Suppose a sequence $\alpha$ is a subsequence of $s$, the support of the sequence $\alpha$ in the sequence database $S$ is the number of tuples in the database containing $\alpha$. The support is defined as:

\begin{equation} \label{eq:support}
    support_{S}(\alpha) = |\{\langle sid, s \rangle|(\langle sid, s \rangle \in S) \wedge (\alpha \subseteq s)\}|
\end{equation}

In this study, the Python library prefixspan \footnote{\url{https://pypi.org/project/prefixspan/}} was utilized to conduct sequential pattern mining. The input for the mining process was the semantic trajectories within each cluster, treated as the sequence database. Regarding the minimum support threshold, it was generally set to 3 for most trajectory clusters. However, for some time spans with only a few constructed trajectories, such as nighttime, a threshold of 1 was employed to ensure the detection of sequences. For each trajectory cluster, the output consisted of several sets of frequently visited places in sequential order, accompanied by their corresponding support values. To identify the typical trajectories, this study selected the sequences with the maximum support and length, and subsequently investigated the places within these trajectories across three dimensions.



\clearpage


% ============================================ ||| Results ||| ============================================
\section{Results}

% ============================= || Spatiotemporal Patterns of Hotspots || =============================
\subsection{Spatiotemporal Patterns of Hotspots}

To investigate the popularity of different areas among locals and tourists at various times, Foursquare check-ins were utilized to detect spatiotemporal hotspots. Figure~\ref{fig:foursquare_checkins_count} and Figure~\ref{fig:foursquare_checkins_trend} reveal the number and temporal pattern of check-ins. Locals exhibit higher activity levels compared to tourists, generating a greater number of check-ins throughout different time periods. Notably, daytime check-ins outnumber nighttime check-ins. Analyzing the temporal pattern, locals display three prominent spikes in check-ins during the daytime, specifically at 8 am, 12 pm, and 6 pm. Conversely, their activities diminish during the nighttime, particularly at midnight. On the other hand, tourists are more active at 12 pm during the daytime but show reduced activity after 6 pm. Furthermore, weekdays experience a higher number of check-ins compared to weekends. Locals exhibit a consistent sharing pattern from Monday to Thursday, with a notable spike on Fridays. Conversely, during weekends, locals share significantly fewer check-ins. Tourists display a similar pattern from Monday to Thursday but exhibit increased activities on Fridays and Saturdays, followed by fewer check-ins on Sundays.

\begin{figure}[!h]
\centering
\includegraphics[width=0.75\textwidth]{figures/foursquare_checkins_count.png}
\caption{\label{fig:foursquare_checkins_count}Number of Foursquare check-ins of locals and tourists across time spans.}
\end{figure}

\begin{figure}[!h]

\begin{subfigure}{0.5\textwidth}
\includegraphics[width=1\linewidth]{figures/foursquare_trend_day.png} 
\caption{Day pattern.}
\label{fig:kde_locals_weekday}
\end{subfigure}
\begin{subfigure}{0.5\textwidth}
\includegraphics[width=1\linewidth]{figures/foursquare_trend_week.png}
\caption{Week pattern.}
\label{fig:kde_tourists_weekday}
\end{subfigure}

\caption{Temporal pattern of Foursquare check-ins.} \label{fig:foursquare_checkins_trend}
\end{figure}


% ====================== | Hotspots of Locals and Tourists | ======================
\subsubsection{Hotspots of Locals and Tourists} \label{hotspots}

% ============== Daytime vs. Nighttime ==============
\subsubsubsection{Daytime vs. Nighttime}
% daytime
Figure~\ref{fig:kde_daytime} illustrates the estimated distribution of Foursquare check-ins during the daytime using \acrshort{kde}. Despite generating fewer check-ins compared to locals, tourists exhibit similar hotspot distribution patterns. The city center, specifically the vicinity surrounding the junction of Westminster, Camden, and the City of London, emerges as a highly popular area among both locals and tourists. These boroughs contain an abundance of cultural and historical landmarks, including the Tower of London, Westminster Abbey, and St. Paul's Cathedral. Moreover, the high transport connectivity of these boroughs enhances their attractiveness. Major train stations, including Waterloo Station and King's Cross St. Pancras, provide convenient access to and from different areas of London. As a result, a significant number of individuals are drawn to these boroughs, leading to a substantial influx of people. Another notable hotspot is situated around Stratford in Newham borough, where the Stratford Shopping Centre is located, which appeals to individuals who enjoy shopping. Additionally, the southern region of Hillingdon, where Heathrow Airport is located, is another hotspot where people share a large number of check-ins. Furthermore, the boundary of Richmond upon Thames and Kingston upon Thames reveals an aggregation of check-ins, and these boroughs contain thriving communities with a blend of history and natural beauty. One notable difference between locals and tourists is the presence of work-related hotspots for locals. The City of London, in particular, serves as a major business hub, with numerous corporate headquarters attracting locals for employment purposes. Similarly, Canary Wharf, located near the Isle of Dogs within the Tower Hamlets borough, represents another hotspot for locals due to its status as part of London's central business district, drawing a large number of professionals. Moreover, boroughs located on the outskirts of London, such as Croydon, Merton, Bromley, Brent, and Harrow, form smaller yet noteworthy hotspots. These areas are characterized by their residential settlement and exhibit a greater concentration of locals compared to tourists.


\begin{figure}[!h]

\begin{subfigure}{0.5\textwidth}
\includegraphics[width=1\linewidth]{figures/kde_locals_daytime.png} 
\caption{Locals.}
\label{fig:kde_locals_daytime}
\end{subfigure}
\begin{subfigure}{0.5\textwidth}
\includegraphics[width=1\linewidth]{figures/kde_tourists_daytime.png}
\caption{Tourists.}
\label{fig:kde_tourists_daytime}
\end{subfigure}

\caption{Kernel density estimation of Foursquare check-ins during the daytime.} \label{fig:kde_daytime}
\end{figure}

% nighttime & comparison
During the nighttime, the city center still serves as the primary hotspot for both locals and tourists, as shown in Figure~\ref{fig:kde_nighttime}. In addition to its cultural and historical attractions, the city center also offers abundant shopping and entertainment options, including renowned areas such as Oxford Street and Regent Street, which provide vibrant nightlife scenes. For locals, the concentration of check-ins in the city center during the nighttime is lower compared to the daytime. Furthermore, the hotspot around Canary Wharf, a bustling business district, diminishes, indicating a reduced number of locals visiting this area during the nighttime. However, the hotspots in Heathrow Airport and boroughs on the outskirts of London, such as Croydon, Bromley, and Harrow, remain prominent. This suggests that these areas are aggregated with a significant number of locals, possibly due to air transport activities ad residential settlements. In terms of locals, although the primary hotspot remains in the city center, the concentration of check-ins in this area decreases during the nighttime. Moreover, there is a reduction or disappearance of hotspots in the suburban areas, indicating a decrease in the number of tourists visiting these regions during the nighttime. However, the hotspot around Heathrow Airport remains stable, suggesting that a certain number of tourists still visit the airport during the nighttime.


\begin{figure}[!h]

\begin{subfigure}{0.5\textwidth}
\includegraphics[width=1\linewidth]{figures/kde_locals_nighttime.png} 
\caption{Locals.}
\label{fig:kde_locals_nighttime}
\end{subfigure}
\begin{subfigure}{0.5\textwidth}
\includegraphics[width=1\linewidth]{figures/kde_tourists_nighttime.png}
\caption{Tourists.}
\label{fig:kde_tourists_nighttime}
\end{subfigure}

\caption{Kernel density estimation of Foursquare check-ins during the nighttime.} \label{fig:kde_nighttime}
\end{figure}


% ============== Weekday vs. Weekend ==============
\subsubsubsection{Weekday vs. Weekend}
% weekday
Figure~\ref{fig:kde_weekday} illustrates the distinct primary hotspots of locals and tourists in the city center during weekdays, though with differences in their spatial distribution. Both groups continue to gravitate towards the area surrounding the junction of Westminster, Camden, and the City of London. However, locals exhibit a relatively lower concentration of check-ins in this area compared to tourists. The City of London, with its prominent financial district and iconic structures such as 30 St Mary Axe and Canary Wharf, serves as an important hub for weekday work engagements, attracting locals to this area. Consistent with the hotspot distribution observed during the daytime, locals also visit Heathrow Airport in the southern region of Hillingdon and various outskirt boroughs such as Croydon, Merton, Bromley, Harrow, Richmond upon Thames, and Kingston upon Thames. In contrast, tourists' primary hotspot in the city center gravitates towards the vicinity of Westminster and Camden during weekdays. These areas contain numerous tourist attractions, shopping destinations, and entertainment venues, making them highly appealing to tourists. Notably, tourists are less active in the outer regions of London during weekdays, indicating a relatively lower level of engagement in these areas. Among the outskirt boroughs, Hillingdon, where Heathrow Airport is located, gains more popularity among tourists compared to other areas.


\begin{figure}[!h]

\begin{subfigure}{0.5\textwidth}
\includegraphics[width=1\linewidth]{figures/kde_locals_weekday.png} 
\caption{Locals.}
\label{fig:kde_locals_weekday}
\end{subfigure}
\begin{subfigure}{0.5\textwidth}
\includegraphics[width=1\linewidth]{figures/kde_tourists_weekday.png}
\caption{Tourists.}
\label{fig:kde_tourists_weekday}
\end{subfigure}

\caption{Kernel density estimation of Foursquare check-ins during weekdays.} \label{fig:kde_weekday}
\end{figure}

% weekend & comparison
During weekends, Figure~\ref{fig:kde_weekend} also shows the scattered pattern of check-ins by locals across the outskirts of London, with a decrease in activity density within the city center. Conversely, tourists concentrate their activities within the inner part of London, sharing fewer check-ins in the outer boroughs. Compared to the weekday hotspots, both locals and tourists exhibit a decrease in check-in density during weekends. Locals demonstrate greater activity in the outskirt boroughs, while tourists focus their attention on the city center.

\begin{figure}[!h]

\begin{subfigure}{0.5\textwidth}
\includegraphics[width=1\linewidth]{figures/kde_locals_weekend.png} 
\caption{Locals.}
\label{fig:kde_locals_weekend}
\end{subfigure}
\begin{subfigure}{0.5\textwidth}
\includegraphics[width=1\linewidth]{figures/kde_tourists_weekend.png}
\caption{Tourists.}
\label{fig:kde_tourists_weekend}
\end{subfigure}

\caption{Kernel density estimation of Foursquare check-ins during weekends.} \label{fig:kde_weekend}
\end{figure}


% ====================== | Mixture of Locals and Tourists in Hotspots | ======================
\subsubsection{Mixture of Locals and Tourists in Hotspots}

% ============== Daytime vs. Nighttime ==============
\subsubsubsection{Daytime vs. Nighttime}
% daytime
The difference ratio provides a measure of the composition of locals and tourists within each raster. A higher difference ratio indicates a larger proportion of tourists, while a lower difference ratio suggests a higher concentration of locals. Figure~\ref{fig:raster_diff_daytime} illustrates the difference ratios between locals and tourists during the daytime. While most areas receive similar popularity from both groups, the city center exhibits distinct distribution patterns for locals and tourists. Tourists tend to cluster around Westminster, Camden, and the City of London, while locals are more concentrated in the eastern region. These findings align with the observations in Section \ref{hotspots}. Heathrow Airport also emerges as a hotspot with a higher activity density of tourists, as reflected by positive difference ratios in certain rasters. The category distribution of check-ins within rasters with significant difference ratios (greater than 0.01 or less than -0.01) during the daytime is represented in Figure~\ref{fig:diff_pop_category_daytime}. Tourists demonstrate a preference for restaurants, transportation, and shopping places, while locals show a higher inclination towards professional places, restaurants, and entertainment places. For more detailed information, please refer to Table~\ref{tab:popular_venues_touristspop_daytime} and Table~\ref{tab:popular_venues_localspop_daytime} in Appendix \ref{appendix}, which highlight the top 10 popular venues within rasters with notable difference ratios during the daytime. In areas with a higher density of tourists, notable venues include transportation hubs like London King's Cross Railway Station, luxury department stores like Harrods, and prominent public squares such as Picadilly Circus and Trafalgar Square. Conversely, in areas with a higher density of locals, professional venues like Google Campus - London and other companies, as well as restaurants around Shoreditch, are more popular.

% nighttime & comparison
Moving to the nighttime, Figure~\ref{fig:raster_diff_nighttime} displays the difference ratios between locals and tourists. The city center, especially Westminster, becomes increasingly popular among tourists, while the area around the City of London shows a balanced presence of locals and tourists, indicating reduced activities of locals during the nighttime. Table~\ref{tab:diff_pop_category_nighttime} reveals that tourists exhibit a heightened interest in entertainment places during the nighttime while maintaining their preference for restaurants and transportation. Locals, on the other hand, shift their focus from professional places to entertainment places while still showing interest in restaurants. Notably, popular nighttime venues for both locals and tourists include nightclubs and pubs, alongside transportation hubs (Table~\ref{tab:popular_venues_touristspop_nighttime}, Table~\ref{tab:popular_venues_localspop_nighttime}).


\begin{figure}[!h]

\begin{subfigure}{0.5\textwidth}
\includegraphics[width=1\linewidth]{figures/raster_diff_daytime.png} 
\caption{Daytime.}
\label{fig:raster_diff_daytime}
\end{subfigure}
\begin{subfigure}{0.5\textwidth}
\includegraphics[width=1\linewidth]{figures/raster_diff_nighttime.png}
\caption{Nighttime.}
\label{fig:raster_diff_nighttime}
\end{subfigure}

\caption{Difference ratio of rasters during the daytime and nighttime.} \label{fig:raster_diff_day}
\end{figure}


\begin{figure}[!h]

\begin{subfigure}{0.5\textwidth}
\includegraphics[width=1\linewidth]{figures/diff_pop_category_daytime.png} 
\caption{Daytime.}
\label{fig:diff_pop_category_daytime}
\end{subfigure}
\begin{subfigure}{0.5\textwidth}
\includegraphics[width=1\linewidth]{figures/diff_pop_category_nighttime.png}
\caption{Nighttime.}
\label{fig:diff_pop_category_nighttime}
\end{subfigure}

\caption{Number of check-in categories in popular areas during the daytime and nighttime.} \label{fig:hotspots_category_day}
\end{figure}



% ============== Weekday vs. Weekend ==============
\subsubsubsection{Weekday vs. Weekend}
% weekday
In Figure~\ref{fig:raster_diff_weekday}, the difference ratio between locals and tourists during weekdays highlights distinct distribution patterns in the city center. As discussed in previous sections, tourists concentrate their activities in Westminster and Camden, while locals exhibit a higher activity density towards the east, particularly around the City of London, a prominent business district. The category distribution of check-ins in rasters with significant difference ratios, predominantly located in the city center, is depicted in Figure~\ref{fig:diff_pop_category_weekday}. Tourists display a preference for transportation, restaurants, professional places, entertainment places, and shopping places, with transportation hubs, public squares, and luxury department stores ranking among their frequently visited venues (Table~\ref{tab:popular_venues_touristspop_weekday}). Similarly, locals also tend to visit professional places, restaurants, transportation, and entertainment places. Popular venues for locals include transportation hubs, technology and financial companies, as well as coffee shops in the vicinity of Shoreditch (Table~\ref{tab:popular_venues_localspop_weekday}).


% weekend & comparison
Moving on to weekends, Figure~\ref{fig:raster_diff_weekend} shows higher difference ratios in the tourist hotspot in the city center, indicating a greater concentration of tourists compared to weekdays. The hotspot of locals in the City of London shows a reduced proportion, suggesting a decrease in local visits to this area during weekends. Figure~\ref{fig:diff_pop_category_weekend} represents the category distribution of check-ins within rasters exhibiting significant difference ratios. Despite a continued presence of tourists in restaurants, transportation, and entertainment places during weekends, their interest in professional places decreases, while interest in shopping places rises. Similarly, locals show reduced visits to professional places but exhibit an inclination towards entertainment places and restaurants in the city center. Regarding popular venues, tourists maintain consistent preferences between weekdays and weekends, while locals shift towards more frequent visits to restaurants, coffee shops, and bars during weekends, rather than company establishments.



\begin{figure}[!h]

\begin{subfigure}{0.5\textwidth}
\includegraphics[width=1\linewidth]{figures/raster_diff_weekday.png} 
\caption{Weekday.}
\label{fig:raster_diff_weekday}
\end{subfigure}
\begin{subfigure}{0.5\textwidth}
\includegraphics[width=1\linewidth]{figures/raster_diff_weekend.png}
\caption{Weekend.}
\label{fig:raster_diff_weekend}
\end{subfigure}

\caption{Difference ratio of rasters during weekdays and weekends.} \label{fig:raster_diff_week}
\end{figure}


\begin{figure}[!h]

\begin{subfigure}{0.5\textwidth}
\includegraphics[width=1\linewidth]{figures/diff_pop_category_weekday.png} 
\caption{Weekday.}
\label{fig:diff_pop_category_weekday}
\end{subfigure}
\begin{subfigure}{0.5\textwidth}
\includegraphics[width=1\linewidth]{figures/diff_pop_category_weekend.png}
\caption{Weekend.}
\label{fig:diff_pop_category_weekend}
\end{subfigure}

\caption{Number of check-in categories in popular areas during weekdays and weekends.} \label{fig:hotspots_category_week}
\end{figure}


% ============================= || Spatiotemporal Patterns of Places || =============================
\subsection{Spatiotemporal Patterns of Places}

% ====================== | Place Distribution | ======================
\subsubsection{Place Distribution}
Places in this study refer to clusters of check-ins. Table~\ref{tab:places_summary} provides an overview of the number of places visited by locals and tourists at different time spans. It is worth noting that the removal of check-in outliers during the clustering process resulted in a reduced number of users compared to the initial value in Table~\ref{tab:foursquare_summary}. Although the number of tourists is considerably greater than that of locals, tourists tend to visit fewer places overall. Figure~\ref{fig:places_distribution_day} displays the distribution of places throughout the day. During daytime hours, a total of 1,525 places are visited by locals, with a scattered distribution across London. While the majority of these places are concentrated in the city center, there are also a certain number of places located in outskirts boroughs such as Hillingdon, Croydon, and Barnet. On the other hand, the 1,046 places visited by tourists exhibit a more concentrated distribution in the inner part of London, with a sparser distribution in the outer areas. During nighttime hours, the distribution of places follows a similar pattern to that of the daytime, but with a lower density. The number of places visited by locals decreases to 598, while tourists visit 441 places, indicating a decrease in overall activity during nighttime hours.


\begin{table}[h!]
\centering
\caption{\label{tab:places_summary}Summary of places.}
\begin{tabular}{llll} \hline
Time & Population group & No. of users & No. of places \\
\hline
\multirow{3}{*}{Daytime} 
& All Users & 6,919 & 2,571 \\
& Locals & 1,077 & 1,525 \\
& Tourists & 5,842 & 1,046 \\
\hline
\multirow{3}{*}{Nighttime} 
& All Users & 4,425 & 1,039 \\
& Locals & 996 & 598 \\
& Tourists & 3,429 & 441 \\
\hline
\multirow{3}{*}{Weekday} 
& All Users & 6,497 & 2,524 \\
& Locals & 1,069 & 1,469 \\
& Tourists & 5,428 & 1,055 \\
\hline
\multirow{3}{*}{Weekend} 
& All Users & 4,886 & 1,435 \\
& Locals & 1,011 & 855 \\
& Tourists & 3,875 & 580 \\
\hline
\end{tabular}
\end{table}


\begin{figure}[!h]

\centering
\begin{subfigure}{0.6\textheight}
\centering
\includegraphics[width=0.9\linewidth]{figures/places_daytime.png} 
\caption{Daytime.}
\label{fig:places_daytime}
\end{subfigure}
\begin{subfigure}{0.6\textheight}
\centering
\includegraphics[width=0.9\linewidth]{figures/places_nighttime.png}
\caption{Nighttime.}
\label{fig:places_nighttime}
\end{subfigure}

\caption{Distribution of places during the daytime and nighttime.}
\label{fig:places_distribution_day}
\end{figure}

Figure~\ref{fig:places_distribution_week} showcases the distribution of places during weekdays and weekends. On weekdays, both locals and tourists visit a significant number of places in the city center, but locals tend to visit more places in suburban areas compared to tourists. Specifically, locals visit 1,469 places, whereas tourists visit 1,055 places. On weekends, the distribution of places for locals and tourists becomes sparser compared to weekdays, but the concentration in the city center remains evident. Similar to weekdays, locals continue to visit more places in the outskirts of London. Overall, the analysis of place distribution reveals distinct patterns in terms of time spans, like daytime versus nighttime, weekday versus weekend, and the spatial concentration of places visited by locals and tourists.


\begin{figure}[!h]

\centering
\begin{subfigure}{0.6\textheight}
\centering
\includegraphics[width=0.9\linewidth]{figures/places_weekday.png}
\caption{Weekday.}
\label{fig:places_weekday}
\end{subfigure}
\begin{subfigure}{0.6\textheight}
\centering
\includegraphics[width=0.9\linewidth]{figures/places_weekend.png}
\caption{Weekend.}
\label{fig:places_weekend}
\end{subfigure}

\caption{Distribution of places during weekdays and weekends.}
\label{fig:places_distribution_week}
\end{figure}



% ====================== | Place Dimensions | ======================
\subsubsection{Place Dimensions}

% ============== Location ==============
\subsubsubsection{Location}
Figure~\ref{fig:places_location_day} provides insight into the distribution of places in the \textit{Location} dimension during the daytime and nighttime. Notably, Westminster and Camden, situated in the city center, emerge as the boroughs with the highest number of places visited by both locals and tourists throughout the day, which also validates the findings in the previous section. Figure~\ref{fig:places_location_daytime} shows that during daytime hours, in most outskirts boroughs, locals tend to visit more places than tourists. This trend is obvious in boroughs such as Barnet, Islington, Harrow, and Merton, as these outer areas serve as residential suburban settlements. It is worth mentioning that Hillingdon, also an outer borough, exhibits a higher number of places visited by tourists, which can be attributed to the presence of Heathrow Airport in its southern region. Boroughs with a relatively equal number of places visited by locals and tourists are predominantly located in the inner part of London. This group includes Westminster, Camden, the City of London, Kensington and Chelsea, and Tower Hamlets, all of which boast a higher concentration of tourist attractions with convenient accessibility. Figure~\ref{fig:places_location_nighttime} illustrates the \textit{Location} dimension of places during the nighttime. While the overall number of places visited decreases in all boroughs during night hours, the distribution remains similar to that of the daytime.


\begin{figure}[!h]

\centering
\begin{subfigure}{0.6\textheight}
\centering
\includegraphics[width=0.9\linewidth]{figures/places_location_daytime.png} 
\caption{Daytime.}
\label{fig:places_location_daytime}
\end{subfigure}
\begin{subfigure}{0.6\textheight}
\centering
\includegraphics[width=0.9\linewidth]{figures/places_location_nighttime.png}
\caption{Nighttime.}
\label{fig:places_location_nighttime}
\end{subfigure}

\caption{Location dimension of places during the daytime and nighttime.}
\label{fig:places_location_day}
\end{figure}

Figure~\ref{fig:places_location_week} presents the distribution of places in the \textit{Location} dimension throughout the week. Generally, there is a decrease in the number of places across various boroughs during weekends, while the relative relationships among boroughs remain consistent, Westminster and Camden consistently emerge as the most popular boroughs among both locals and tourists throughout the entire week. It is noteworthy that while most boroughs have a higher number of places visited by locals, Westminster, Kensington and Chelsea, and Hillingdon attract tourists to visit an equal or greater number of places. The former two boroughs, located in inner London, boast numerous tourist attractions and iconic architecture, making them appealing to both locals and tourists. On the other hand, Hillingdon, situated on the outskirts of London and home to Heathrow Airport, draws a large number of tourists to visit. The analysis of the \textit{Location} dimension of places provides valuable insights into the spatial patterns of places across different time spans, further highlighting the concentration of places in key boroughs and the varying preferences of locals and tourists across different parts of London.

\begin{figure}[!h]

\centering
\begin{subfigure}{0.6\textheight}
\centering
\includegraphics[width=0.9\linewidth]{figures/places_location_weekday.png}
\caption{Weekday.}
\label{fig:places_location_weekday}
\end{subfigure}
\begin{subfigure}{0.6\textheight}
\centering
\includegraphics[width=0.9\linewidth]{figures/places_location_weekend.png}
\caption{Weekend.}
\label{fig:places_location_weekend}
\end{subfigure}

\caption{Location dimension of places during weekdays and weekends.}
\label{fig:places_location_week}
\end{figure}




% ============== Locale ==============
\subsubsubsection{Locale}
Figure~\ref{fig:places_locale_day} provides insights into the distribution of places in the \textit{Locale} dimension throughout the day. During the daytime, the most visited categories for both locals and tourists include restaurants, professional places, shopping places, entertainment places, and transport places. Generally, locals visit more places across most categories, but there are a few exceptions. Tourists show a higher preference for places in the accommodation and green \& blue space categories. At nighttime, restaurants, entertainment places, and transport places remain popular categories for both locals and tourists, while there is a noticeable decrease in the number of places in professional places and shopping places compared to daytime hours. The difference between locals and tourists in shopping places diminishes, suggesting that locals tend to explore fewer shopping places during the nighttime. Accommodation and green \& blue space continue to attract more tourists than locals during nighttime hours. Notably, the difference between the number of places visited by locals and tourists in the accommodation category increases, indicating an amplified demand for accommodating among tourists during the nighttime.


\begin{figure}[!h]

\centering
\begin{subfigure}{0.6\textheight}
\centering
\includegraphics[width=0.9\linewidth]{figures/places_locale_daytime.png} 
\caption{Daytime.}
\label{fig:places_locale_daytime}
\end{subfigure}
\begin{subfigure}{0.6\textheight}
\centering
\includegraphics[width=0.9\linewidth]{figures/places_locale_nighttime.png}
\caption{Nighttime.}
\label{fig:places_locale_nighttime}
\end{subfigure}

\caption{Locale dimension of places during the daytime and nighttime.}
\label{fig:places_locale_day}
\end{figure}



Figure~\ref{fig:places_locale_week} shows the distribution of places in the \textit{Locale} dimension throughout the week. On weekdays, restaurants, professional places, entertainment places, shopping places, and transport places are the most popular categories for both locals and tourists. Categories such as art places, green \& blue space, others, service places, and sports places have fewer places visited, and locals show a greater inclination to explore these categories compared to tourists. However, when it comes to the accommodation category, tourists tend to visit more places in this category than locals. The popular categories during weekends remain consistent with those of weekdays, with restaurants, entertainment places, shopping places, and transport places retaining their popularity. However, professional places experience a decrease in visitation, along with four other categories. Analyzing the difference between locals and tourists in various categories, it is noticeable that tourists tend to visit a relatively higher number of places than locals in the accommodation category. Conversely, locals exhibit a greater preference for categories like green \& blue space and sports places, visiting more places in these categories compared to tourists.


\begin{figure}[!h]

\centering
\begin{subfigure}{0.6\textheight}
\centering
\includegraphics[width=0.9\linewidth]{figures/places_locale_weekday.png}
\caption{Weekday.}
\label{fig:places_locale_weekday}
\end{subfigure}
\begin{subfigure}{0.6\textheight}
\centering
\includegraphics[width=0.9\linewidth]{figures/places_locale_weekend.png}
\caption{Weekend.}
\label{fig:places_locale_weekend}
\end{subfigure}

\caption{Locale dimension of places during weekdays and weekends.}
\label{fig:places_locale_week}
\end{figure}


% ============== Sense of Place ==============
\subsubsubsection{Sense of Place}
Figure~\ref{fig:places_topics_sense_locals_day} shows the distribution of locals' places in the \textit{Sense of Place} dimension, along with the corresponding word clouds for topics generated from Flickr tags throughout the day. During the daytime, Topic 0 and Topic 4 are prominent, with over 700 places associated with air and rail transport (e.g., lhr\footnote{The airport London Heathrow.}, airbus, bus, railway in Figure~\ref{fig:topics_daytime_locals}). However, there are also some topics difficult to interpret, such as Topic 5, which contains fewer than 50 places. This topic is characterized by popular tags like people and candid, making it difficult to understand how people describe these particular places. Figure~\ref{fig:topics_distribution_daytime_locals} in Appendix \ref{appendix} shows the distribution of locals' places across various topics during the daytime. Generally, the distribution aligns with the topic content. For instance, Topic 1 (e.g., park, museum, spring) and Topic 2 (e.g., june, party, summer) mainly encompass entertainment, and areas like Westminster and Camden exhibit a high concentration of places associated with these topics. Nevertheless, some inconsistencies exist between the topic distribution and content. Topic 0, primarily related to air transport, displays places both around Heathrow Airport and in the city center due to its coverage of other city activities (e.g., ship, regentscanal, market). Topic 4, linked to rail transport, has its places distributed across the city center and outskirts boroughs. Topic 6, centered around the Olympics, exhibits a concentration of places near London Stadium in Stratford, Newham (built for the 2012 Olympics), but also includes locations unrelated to the topic.

During the nighttime, Topic 0 (paralympics, crowd, cycling, performance) emerges as the most popular topic, with approximately 300 associated places. The remaining five topics are less frequently mentioned, each comprising fewer than 100 places. However, these topics also reveal locals' activities during nighttime hours. Contrary to the daytime, nighttime has locals discuss live, music, gig, and concert, indicating a shift towards more leisurely activities. Analyzing the distribution of places within each topic (Figure~\ref{fig:topics_distribution_nighttime_locals}), Topic 0 has more places than others and exhibits a scattered distribution pattern throughout London, primarily concentrated in the city center. Topic 1 has its places mainly distributed in Westminster, known for its vibrant gig and live music venues in the evening. Topic 2 concentrates its places on the boundary of the City of London, Hackney, and Tower Hamlets, encompassing areas like Shoreditch renowned for its nightlife venues like bars and clubs. Topic 3 encompasses tags related to the city, architecture, and Southbank, indicating a connection to art and culture. Places associated with Topic 3 are along the River Thames in the City of London, Westminster, Lambeth, and Southwark. Topic 4 and Topic 5 also contain tags related to nighttime activities but display a more dispersed distribution of places.

\begin{figure}[!h]
    \centering
    \begin{subfigure}{0.45\textwidth}
        \centering
        \includegraphics[width=\linewidth]{figures/places_sense_daytime_locals.png} 
        \caption{Frequency of topics during the daytime.}
        \label{fig:places_sense_daytime_locals}
    \end{subfigure}
    \hfill
    \begin{subfigure}{0.5\textwidth}
        \centering
        \includegraphics[width=\linewidth]{figures/topics_daytime_locals.png} 
        \caption{Word clouds for topics during the daytime.}
        \label{fig:topics_daytime_locals}
    \end{subfigure}

    \begin{subfigure}{0.45\textwidth}
        \centering
        \includegraphics[width=\linewidth]{figures/places_sense_nighttime_locals.png}
        \caption{Frequency of topics during the nighttime.}
        \label{fig:places_sense_nighttime_locals}
    \end{subfigure}
    \hfill
    \begin{subfigure}{0.5\textwidth}
        \centering
        \includegraphics[width=\linewidth]{figures/topics_nighttime_locals.png}
        \caption{Word clouds for topics during the nighttime.}
        \label{fig:topics_nighttime_locals}
    \end{subfigure}

    \caption{Sense of place dimension of locals' places during the daytime and nighttime.}
    \label{fig:places_topics_sense_locals_day}
\end{figure}

Compared to locals, tourists demonstrate a preference for urban-related places and the Olympics during the daytime (Figure~\ref{fig:places_sense_daytime_tourists} and Figure~\ref{fig:topics_daytime_tourists}). Among places visited by tourists, over 350 places are associated with Topic 4 (e.g., street, bus, city), and over 250 places are associated with Topic 0 (e.g., olympics, paralympics). However, places related to air transport are less popular among tourists, as Topic 1 (e.g., lhr, aircraft, airbus) only contains about 10 places of tourists. Analyzing the distribution of tourists' places across various topics (Figure~\ref{fig:topics_distribution_daytime_tourists}), places associated with the Olympics (Topic) 0 cover not only areas around London Stadium but also the city center like Westminster, Camden. Topic 1 predominantly includes places near Heathrow Airport, aligning with the topic content. Topic 2 describes places with tags like tower, city, bridge, and river, and the associated places are located along the bustling part of the River Thames with many bridges. Places linked to Topic 3 (e.g., marathon, city, urban) exhibit a distribution pattern along the River Thames, concentrated in the City of London, similar to the route of the London 2012 Olympic Marathon. Boroughs such as the City of London, Westminster, and Kensington and Chelsea are representative of urban life and boast art venues like museums and sculpture, thus, places in these boroughs are described by Topic 4 and Topic 5.

Regarding the topics of places during nighttime hours, although the total number of places visited by tourists decreases, there is relatively increased interest in places related to museums and transport compared to the daytime, as evidenced by the prominence of Topic 0 (e.g., museum, lhr, station, railway). Similar to locals, tourists during the nighttime also show interest in places related to nighttime activities, reflected by the emergence of Topic 1 (e.g., concert, live, gig, music) and Topic 4 (nighttime, city). Figure~\ref{fig:topics_distribution_nighttime_tourists} displays the distribution of tourists' places across various topics during the nighttime. While both locals and tourists visit places related to gigs and live music, tourists tend to prefer places around Shoreditch rather than Westminster, as indicated by the distribution of places in Topic 1. Similar to locals, tourists also enjoy visiting places near bridges along the River Thames, which are described by Topic 2.


\begin{figure}[!h]
    \centering
    \begin{subfigure}{0.45\textwidth}
        \centering
        \includegraphics[width=\linewidth]{figures/places_sense_daytime_tourists.png} 
        \caption{Frequency of topics during the daytime.}
        \label{fig:places_sense_daytime_tourists}
    \end{subfigure}
    \hfill
    \begin{subfigure}{0.5\textwidth}
        \centering
        \includegraphics[width=\linewidth]{figures/topics_daytime_tourists.png} 
        \caption{Word clouds for topics during the daytime.}
        \label{fig:topics_daytime_tourists}
    \end{subfigure}

    \begin{subfigure}{0.45\textwidth}
        \centering
        \includegraphics[width=\linewidth]{figures/places_sense_nighttime_tourists.png}
        \caption{Frequency of topics during the nighttime.}
        \label{fig:places_sense_nighttime_tourists}
    \end{subfigure}
    \hfill
    \begin{subfigure}{0.5\textwidth}
        \centering
        \includegraphics[width=\linewidth]{figures/topics_nighttime_tourists.png}
        \caption{Word clouds for topics during the nighttime.}
        \label{fig:topics_nighttime_tourists}
    \end{subfigure}

    \caption{Sense of place dimension of tourists' places during the daytime and nighttime.}
    \label{fig:places_topics_sense_tourists_day}
\end{figure}

Figure~\ref{fig:places_topics_sense_locals_week} illustrates the distribution of locals' places in the \textit{Sense of Place} dimension throughout the week. On weekdays, locals visit places that can be categorized into five topics with three themes: transport-related (Topic 0), entertainment-related (Topic 1), and cityscape-related (Topic 2, Topic 3, Topic 4). Topic 0 stands out with nearly 700 associated places, while the other four topics contain around 200 places each. The distribution of places reveals that Topic 0 covers various areas of London, with a higher concentration in the city center due to the accessibility of transport facilities. Places linked to Topic 1 can be found in Lambeth, Kensington and Chelsea, Westminster, Camden, and the City of London, described by tags like live, gig, and southbank. Places linked to Topic 2 and Topic 4 display a distribution pattern primarily within the inner part of London, whereas Topic 3 also includes a significant number of places in the outer regions of the city (Figure~\ref{fig:topics_distribution_weekday_locals}).

During weekends, locals show a shift in their preferences towards more relaxing places, as indicated by the popularity of Topic 0 (e.g., park, garden) and Topic 1 (e.g., graffiti, streetart, station), each comprising over 250 places. Additionally, locals explore sports-related places represented by Topic 4 (e.g., city, gbr, football, race) and Topic 5 (e.g., paralympics, olympics). The distribution of places in these topics demonstrates dispersion, as parks and graffiti can be found throughout the city. Places associated with Topic 3 concentrate in the city center where architectural wonders are abundant. Topic 4 shows a similar trend with its places located around sports venues like Emirates Stadium, which is a football stadium. Topic 5 encompasses places related to the Olympics, bridges, and the Shard, distributed around London Stadium in Stratford and along the River Thames (Figure~\ref{fig:topics_distribution_weekend_locals}).

\begin{figure}[!h]
    \centering
    \begin{subfigure}{0.45\textwidth}
        \centering
        \includegraphics[width=\linewidth]{figures/places_sense_weekday_locals.png} 
        \caption{Frequency of topics during weekdays.}
        \label{fig:places_sense_weekday_locals}
    \end{subfigure}
    \hfill
    \begin{subfigure}{0.5\textwidth}
        \centering
        \includegraphics[width=\linewidth]{figures/topics_weekday_locals.png} 
        \caption{Word clouds for topics during weekdays.}
        \label{fig:topics_weekday_locals}
    \end{subfigure}

    \begin{subfigure}{0.45\textwidth}
        \centering
        \includegraphics[width=\linewidth]{figures/places_sense_weekend_locals.png}
        \caption{Frequency of topics during weekends.}
        \label{fig:places_sense_weekdend_locals}
    \end{subfigure}
    \hfill
    \begin{subfigure}{0.5\textwidth}
        \centering
        \includegraphics[width=\linewidth]{figures/topics_weekend_locals.png}
        \caption{Word clouds for topics during weekends.}
        \label{fig:topics_weekend_locals}
    \end{subfigure}

    \caption{Sense of place dimension of locals' places during weekdays and weekends.}
    \label{fig:places_topics_sense_locals_week}
\end{figure}

Figure~\ref{fig:places_topics_sense_tourists_week} illustrates the places visited by tourists in the \textit{Sense of Place} dimension throughout the week. On weekdays, tourists exhibit a preference for places associated with Topic 0 (e.g., olympics, paralympics) and Topic 5 (e.g., street, graffiti, city, art), with each topic comprising approximately 250 places. Tourists also like to visit places related to transport, art, and tourist attractions, as evidenced by the high occurrence of Topic 1 (e.g., museum, victoria, station, art) and Topic 3 (e.g., architecture, bridge, shard), both featuring over 150 associated places. Analyzing the distribution of places across various topics on weekdays, most of these places are concentrated in the inner part of London, while places linked to Topic 0 and Topic 4 also display a small concentration around Heathrow Airport (Figure~\ref{fig:topics_distribution_weekday_locals}).

During weekends, tourists continue to visit places characterized by architecture, street art, rail and air transport, the Olympics, and bridges, similar to their choices on weekdays. Topic 0 (e.g., marathon, city, architecture) encompasses the largest number of tourist destinations, exceeding 250 places, while places associated with Topic 2 (e.g., lhr, airbus, aircraft) are comparatively less popular among tourists, with fewer than 50 places. Concerning the distribution of places across various topics, except for places linked to Topic 0 which are distributed throughout London with a concentration in the city center, Topic 1, Topic 3, and Topic 5 predominantly feature places located around the city center. Topic 2 exhibits a concentration at Heathrow Airport, while Topic 4 is scattered in the northern part of the River Thames.

\begin{figure}[!h]
    \centering
    \begin{subfigure}{0.45\textwidth}
        \centering
        \includegraphics[width=\linewidth]{figures/places_sense_weekday_tourists.png} 
        \caption{Frequency of topics during weekdays.}
        \label{fig:places_sense_weekday_tourists}
    \end{subfigure}
    \hfill
    \begin{subfigure}{0.5\textwidth}
        \centering
        \includegraphics[width=\linewidth]{figures/topics_weekday_tourists.png} 
        \caption{Word clouds for topics during weekdays.}
        \label{fig:topics_weekday_tourists}
    \end{subfigure}

    \begin{subfigure}{0.45\textwidth}
        \centering
        \includegraphics[width=\linewidth]{figures/places_sense_weekend_tourists.png}
        \caption{Frequency of topics during weekends.}
        \label{fig:places_sense_weekdend_tourists}
    \end{subfigure}
    \hfill
    \begin{subfigure}{0.5\textwidth}
        \centering
        \includegraphics[width=\linewidth]{figures/topics_weekend_tourists.png}
        \caption{Word clouds for topics during weekends.}
        \label{fig:topics_weekend_tourists}
    \end{subfigure}

    \caption{Sense of place dimension of tourists' places during weekdays and weekends.}
    \label{fig:places_topics_sense_tourists_week}
\end{figure}

% ============================= || Spatiotemporal City Perception through Semantic Trajectory || =============================
\subsection{Spatiotemporal City Perception through Semantic Trajectory}


% ====================== | Semantic Trajectory Distribution | ======================
\subsubsection{Semantic Trajectory Distribution}


% ====================== | Semantic Trajectory Dimensions | ======================
\subsubsection{Semantic Trajectory Dimensions}


% ====================== | Typical Semantic Trajectory and City Perception | ======================
\subsubsection{Typical Semantic Trajectory and City Perception}

RQ2: How do locals and tourists perceive the city along their semantic trajectories at different time spans



\clearpage


\section{Discussion}
\subsection{Research Question 1}

\subsection{Research Question 2}

\subsection{Limitations}
limitation: foursquare is not for everyone, most users are young people

silhouette scores in trajectory clustering are bad, most of them were negative values
intra-cluster distance

\clearpage


\section{Conclusion}

\clearpage


\pagenumbering{roman}

\bibliographystyle{apacite}
\bibliography{references}

\clearpage


\appendix
\section{Appendix} \label{appendix}
% ========================= top 10 popular venues =========================
\begin{table}[!h]
\centering
\caption{\label{tab:popular_venues_touristspop_daytime}Top 10 popular venues within rasters with a difference ratio exceeding 0.01 (indicating a higher number of tourists) during the daytime.}
\begin{adjustbox}{max width=\textwidth, margin=0cm}
\begin{threeparttable}
\begin{tabular}{lp{5cm}lp{4cm}} \hline
No. & Venue Name & No. of Check-ins & Description \\ \hline
1 & London King's Cross Railway Station (KGX) & 1030 & Transportation hub \\
2 & London Victoria Railway Station (VIC) & 806 & Transportation hub \\
3 & Harrods & 716 & Luxury department store \\
4 & London St Pancras International Railway Station (STP) & 682 & Transportation hub \\
5 & London St Pancras International Eurostar Terminal & 511 & Transportation hub \\
6 & Piccadilly Circus & 473 & Public space with iconic illuminated billboards \\
7 & Trafalgar Square & 465 & Public square with historical and cultural landmarks \\
8 & Selfridges & 464 & Luxury department store \\
9 & Buckingham Palace & 459 & Iconic official residence of the British monarch \\
10 & Big Ben (Elizabeth Tower) & 352 & Iconic tower clock \\ \hline
\end{tabular}
\end{threeparttable}
\end{adjustbox}
\end{table}

\begin{table}[!h]
\centering
\caption{\label{tab:popular_venues_localspop_daytime}Top 10 popular venues within rasters with a difference ratio lower than -0.01 (indicating a higher number of locals) during the daytime.}
\begin{adjustbox}{max width=\textwidth, margin=0cm}
\begin{threeparttable}
\begin{tabular}{lp{5cm}lp{4cm}} \hline
No. & Venue Name & No. of Check-ins & Description \\ \hline
1 & Google Campus - London & 160 & Vibrant hub for tech startups and entrepreneurs \\
2 & Shoreditch Grind & 94 & Trendy coffee shop and cocktail bar \\
3 & Ozone Coffee Roasters & 84 & Renowned specialty coffee roastery \\
4 & Old Street London Underground Station & 84 & Transportation hub \\
5 & SapientRazorfish & 77 & Global digital consultancy \\
6 & Shoreditch House & 61 & Private members' club \\
7 & CCA International & 59 & Global customer experience management company \\
8 & BOXPARK Shoreditch & 56 & Innovative retail and dining destination \\
9 & Dishoom & 50 & Restaurant \\
10 & Shoreditch Triangle & 46 & Cultural and creative hub \\ \hline
\end{tabular}
\end{threeparttable}
\end{adjustbox}
\end{table}


\begin{table}[!h]
\centering
\caption{\label{tab:popular_venues_touristspop_nighttime}Top 10 popular venues within rasters with a difference ratio exceeding 0.01 (indicating a higher number of tourists) during the nighttime.}
\begin{adjustbox}{max width=\textwidth, margin=0cm}
\begin{threeparttable}
\begin{tabular}{lp{5cm}lp{4cm}} \hline
No. & Venue Name & No. of Check-ins & Description \\ \hline
1 & London Euston Railway Station & 199 & Transportation hub \\
2 & Piccadilly Circus & 179 & Public space with iconic illuminated billboards \\
3 & London Paddington Railway Station (PAD) & 165 & Transportation hub \\
4 & London Victoria Railway Station (VIC) & 146 & Transportation hub \\
5 & Trafalgar Square & 115 & Public square with historical and cultural landmarks \\
6 & Charing Cross Railway Station (CHX) & 111 & Transportation hub \\
7 & Heaven & 95 & Nightclub \\
8 & Leicester Square & 91 & Bustling entertainment hub \\
9 & Harrods & 57 & Luxury department store \\
10 & The Harp, Covent Garden & 52 & Pub \\ \hline
\end{tabular}
\end{threeparttable}
\end{adjustbox}
\end{table}

\begin{table}[!h]
\centering
\caption{\label{tab:popular_venues_localspop_nighttime}Top 10 popular venues within rasters with a difference ratio lower than -0.01 (indicating a higher number of locals) during the nighttime.}
\begin{adjustbox}{max width=\textwidth, margin=0cm}
\begin{threeparttable}
\begin{tabular}{lp{5cm}lp{4cm}} \hline
No. & Venue Name & No. of Check-ins & Description \\ \hline
1 & Shoreditch House & 45 & Private members' club \\
2 & Tesco Express & 31 & Convenience store \\
3 & Old Street London Underground Station & 21 & Transportation hub \\
4 & Xoyo & 21 & Nightclub \\
5 & TfL Bus 314 & 21 & Bus service \\
6 & Strongroom 314 & 20 & Recording studio \\
7 & Zigfrid von Underbelly & 19 & Bar \\
8 & Concrete & 19 & Nightclub \\
9 & The Park & 19 & Green space \\
10 & The Old Blue Last & 19 & Pub \\ \hline
\end{tabular}
\end{threeparttable}
\end{adjustbox}
\end{table}


\begin{table}[!h]
\centering
\caption{\label{tab:popular_venues_touristspop_weekday}Top 10 popular venues within rasters with a difference ratio exceeding 0.01 (indicating a higher number of tourists) during weekdays.}
\begin{adjustbox}{max width=\textwidth, margin=0cm}
\begin{threeparttable}
\begin{tabular}{lp{5cm}lp{4cm}} \hline
No. & Venue Name & No. of Check-ins & Description \\ \hline
1 & London Euston Railway Station & 1100 & Transportation hub \\
2 & London King's Cross Railway Station (KGX) & 937 & Transportation hub \\
3 & London Victoria Railway Station (VIC) & 695 & Transportation hub \\
4 & London St Pancras International Railway Station (STP) & 562 & Transportation hub \\
5 & Harrods & 544 & Luxury department store \\
6 & London St Pancras International Eurostar Terminal & 454 & Transportation hub \\
7 & Piccadilly Circus & 446 & Public space with iconic illuminated billboards \\
8 & Trafalgar Square & 341 & Public square with historical and cultural landmarks \\
9 & Selfridges & 329 & Luxury department store \\
10 & Buckingham Palace & 319 & Iconic official residence of the British monarch \\ \hline
\end{tabular}
\end{threeparttable}
\end{adjustbox}
\end{table}

\begin{table}[!h]
\centering
\caption{\label{tab:popular_venues_localspop_weekday}Top 10 popular venues within rasters with a difference ratio lower than -0.01 (indicating a higher number of locals) during weekdays.}
\begin{adjustbox}{max width=\textwidth, margin=0cm}
\begin{threeparttable}
\begin{tabular}{lp{5cm}lp{4cm}} \hline
No. & Venue Name & No. of Check-ins & Description \\ \hline
1 & London Liverpool Street Railway Station (LST) & 604 & Transportation hub \\
2 & Google Campus - London & 155 & Vibrant hub for tech startups and entrepreneurs \\
3 & Shoreditch Grind & 84 & Trendy coffee shop and cocktail bar \\
4 & Shoreditch House & 84 & Private members' club \\
5 & Old Street London Underground Station & 83 & Transportation hub \\
6 & SapientRazorfish & 77 & Global digital consultancy \\
7 & UBS Wealth Management & 77 & Global financial service firm \\
8 & Ozone Coffee Roasters & 75 & Renowned specialty coffee roastery \\
9 & Bank London Underground and DLR Station & 73 & Transportation hub \\
10 & Liverpool Street London Underground Station & 72 & Transportation hub \\ \hline
\end{tabular}
\end{threeparttable}
\end{adjustbox}
\end{table}


\begin{table}[!h]
\centering
\caption{\label{tab:popular_venues_touristspop_weekend}Top 10 popular venues within rasters with a difference ratio exceeding 0.01 (indicating a higher number of tourists) during weekends.}
\begin{adjustbox}{max width=\textwidth, margin=0cm}
\begin{threeparttable}
\begin{tabular}{lp{5cm}lp{4cm}} \hline
No. & Venue Name & No. of Check-ins & Description \\ \hline
1 & London Paddington Railway Station (PAD) & 278 & Transportation hub \\
2 & London King's Cross Railway Station (KGX) & 266 & Transportation hub \\
3 & London Victoria Railway Station (VIC) & 257 & Transportation hub \\
4 & Trafalgar Square & 239 & Public square with historical and cultural landmarks \\
5 & London St Pancras International Railway Station (STP) & 229 & Transportation hub \\
6 & Harrods & 229 & Luxury department store \\
7 & Piccadilly Circus & 206 & Public space with iconic illuminated billboards \\
8 & Buckingham Palace & 183 & Iconic official residence of the British monarch \\
9 & Selfridges & 177 & Luxury department store \\
10 & Big Ben (Elizabeth Tower) & 163 & Iconic tower clock \\ \hline
\end{tabular}
\end{threeparttable}
\end{adjustbox}
\end{table}

\begin{table}[!h]
\centering
\caption{\label{tab:popular_venues_localspop_weekend}Top 10 popular venues within rasters with a difference ratio lower than -0.007 (indicating a higher number of locals) during weekends.}
\begin{adjustbox}{max width=\textwidth, margin=0cm}
\begin{threeparttable}
\begin{tabular}{lp{5cm}lp{4cm}} \hline
No. & Venue Name & No. of Check-ins & Description \\ \hline
1 & BOXPARK Shoreditch & 27 & Innovative retail and dining destination \\
2 & Shoreditch Grind & 23 & Trendy coffee shop and cocktail bar \\
3 & Shoreditch House & 22 & Private members' club \\
4 & Old Street London Underground Station & 22 & Transportation hub \\
5 & Hoxton Grill & 15 & Restaurant \\
6 & Shoreditch Triangle & 15 & Cultural and creative hub \\
7 & The Hoxton, Shoreditch & 15 & Hotel \\
8 & The Water Poet & 13 & Pub \\
9 & Zigfrid von Underbelly & 12 & Bar \\
10 & Ozone Coffee Roasters & 12 & Renowned specialty coffee roastery \\ \hline
\end{tabular}
\end{threeparttable}
\end{adjustbox}
\end{table}


% ========================= topic distribution =========================
% locals_day
\begin{figure}[!h]
\centering
\includegraphics[width=0.75\textwidth]{figures/topics_distribution_daytime_locals.png}
\caption{\label{fig:topics_distribution_daytime_locals}Distribution of locals' topics during the daytime.}
\end{figure}

\begin{figure}[!h]
\centering
\includegraphics[width=0.75\textwidth]{figures/topics_distribution_nighttime_locals.png}
\caption{\label{fig:topics_distribution_nighttime_locals}Distribution of locals' topics during the nighttime.}
\end{figure}

% tourists_day
\begin{figure}[!h]
\centering
\includegraphics[width=0.75\textwidth]{figures/topics_distribution_daytime_tourists.png}
\caption{\label{fig:topics_distribution_daytime_tourists}Distribution of tourists' topics during the daytime.}
\end{figure}

\begin{figure}[!h]
\centering
\includegraphics[width=0.75\textwidth]{figures/topics_distribution_nighttime_tourists.png}
\caption{\label{fig:topics_distribution_nighttime_tourists}Distribution of tourists' topics during the nighttime.}
\end{figure}

% locals_week
\begin{figure}[!h]
\centering
\includegraphics[width=0.75\textwidth]{figures/topics_distribution_weekday_locals.png}
\caption{\label{fig:topics_distribution_weekday_locals}Distribution of locals' topics during weekdays.}
\end{figure}

\begin{figure}[!h]
\centering
\includegraphics[width=0.75\textwidth]{figures/topics_distribution_weekend_locals.png}
\caption{\label{fig:topics_distribution_weekend_locals}Distribution of locals' topics during weekends.}
\end{figure}

% tourists_week
\begin{figure}[!h]
\centering
\includegraphics[width=0.75\textwidth]{figures/topics_distribution_weekday_tourists.png}
\caption{\label{fig:topics_distribution_weekday_tourists}Distribution of tourists' topics during weekdays.}
\end{figure}

\begin{figure}[!h]
\centering
\includegraphics[width=0.75\textwidth]{figures/topics_distribution_weekend_tourists.png}
\caption{\label{fig:topics_distribution_weekend_tourists}Distribution of tourists' topics during weekends.}
\end{figure}







boxplot of category


% Figure \ref{fig:flickr_day_count_pie}
% \begin{figure}[!h]
% \centering
% \includegraphics[width=1.13\textwidth]{figures/flickr_day_count_pie.png}
% \caption{\label{fig:flickr_day_count_pie}Percent of Flickr data of locals and tourists during the daytime and nighttime.}
% \end{figure}

% Figure \ref{fig:flickr_trend_day}
% \begin{figure}[!h]
% \centering
% \includegraphics[width=0.8\textwidth]{figures/flickr_trend_day.png}
% \caption{\label{fig:flickr_trend_day}Temporal Flickr photos sharing pattern along the day.}
% \end{figure}

% Figure \ref{fig:flickr_trend_week}
% \begin{figure}[!h]
% \centering
% \includegraphics[width=0.8\textwidth]{figures/flickr_trend_week.png}
% \caption{\label{fig:flickr_trend_week}Temporal Flickr photos sharing pattern along the week.}
% \end{figure}


\end{document}